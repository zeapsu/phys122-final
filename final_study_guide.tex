\documentclass{article}
\usepackage[a4paper, total={6in, 8in}]{geometry}
\usepackage{amsmath}
\usepackage{amssymb}
\usepackage{physics}
\usepackage{mathtools}
\usepackage{enumitem}
\usepackage{tcolorbox}
\usepackage{graphicx}
\usepackage{setspace}
\usepackage{siunitx}
\usepackage{tikz}
\usepackage{pgfplots}
\usepackage{float}

\usepackage{hyperref}
\hypersetup{
    colorlinks=true,
    linkcolor=black,
    filecolor=magenta,      
    urlcolor=blue,
    pdftitle={Modern Physics Final Study Guide},
    pdfpagemode=FullScreen,
}

\usepackage{fancyhdr}
\pagestyle{fancy}
\fancyhf{} % clear all header and footer fields
\fancyhead[L]{\leftmark} % put section name on the left
\fancyhead[R]{\thepage} % page number on the right
\renewcommand{\headrulewidth}{0.4pt} % horizontal line under header
\renewcommand{\sectionmark}[1]{\markboth{\small#1}{}}

% To handle the first page differently
\fancypagestyle{plain}{
    \fancyhf{} % clear all header and footer fields
    \renewcommand{\headrulewidth}{0pt} % no line for first page
}

\onehalfspacing

\pgfplotsset{compat=1.18}
\usetikzlibrary{angles,quotes,arrows.meta}

\AtBeginDocument{\RenewCommandCopy\qty\SI}

% Custom commands for consistent formatting
\newcommand{\eqbox}[1]{\begin{tcolorbox}[colback=gray!10] #1 \end{tcolorbox}}
\newcommand{\conceptbox}[1]{\begin{tcolorbox}[colback=blue!10] #1 \end{tcolorbox}}
\newcommand{\tipbox}[1]{\begin{tcolorbox}[colback=green!10] #1 \end{tcolorbox}}
\newcommand{\warningbox}[1]{\begin{tcolorbox}[colback=red!10] #1 \end{tcolorbox}}

% Constants
\newcommand{\planckconst}{6.626\times 10^{-34}\,\text{J}\cdot\text{s}}
\newcommand{\redplanck}{1.055\times 10^{-34}\,\text{J}\cdot\text{s}}
\newcommand{\electronmass}{9.11\times 10^{-31}\,\text{kg}}
\newcommand{\lightspeed}{3.0\times 10^{8}\,\text{m}/\text{s}}
\newcommand{\bohrradius}{0.0529\times 10^{-9}\,\text{m}}
\newcommand{\stefan}{5.67\times10^{-8}\text{ W/m}^2\cdot K^4}
\newcommand{\compton}{0.002426\text{ nm}}

\title{Modern Physics Final Study Guide}
\author{PHYS-122}
\date{Fall 2024}

\begin{document}
\maketitle

\newpage

\tableofcontents

\newpage

\section{Special Relativity}

\subsection{Key Concepts}

\conceptbox{
\textbf{Postulates:} 
\begin{itemize}
    \item Laws of physics are the same in all inertial frames.
    \item The speed of light is constant in a vacuum for all observers, regardless of their motion relative to the source.
\end{itemize}
}

\conceptbox{
\textbf{Relativity of Simultaneity:} 
Two events that appear simultaneous to one observer may not appear simultaneous to another observer moving relative to the first.
}

\conceptbox{
\textbf{Relativistic Velocity Addition:} 
These problems typically involve three objects:
\begin{itemize}
    \item $S$: Stationary reference frame.
    \item $S'$: Reference frame of the second moving object.
    \item $u$: Velocity of the first object relative to the stationary frame.
    \item $v$: Velocity of the second object relative to the stationary frame.
    \item $u'$: Velocity of the first object relative to the second moving object.
\end{itemize}
}

\conceptbox{
\textbf{The Relativistic Invariant Interval:} 
A spacetime quantity that remains unchanged under Lorentz transformations.

\begin{itemize}
    \item \textbf{Properties:}
    \begin{itemize}
        \item Negative spatial components distinguish it from a Euclidean interval.
        \item It can be positive, negative, or zero:
        \begin{itemize}
            \item \textbf{Time-like ($s^2 > 0$):} Events can be connected by a physical particle moving slower than the speed of light.
            \item \textbf{Space-like ($s^2 < 0$):} Events are separated by more distance than time; no signal can connect them.
            \item \textbf{Light-like ($s^2 = 0$):} Events are connected by a light signal.
        \end{itemize}
    \end{itemize}
    \item \textbf{Relationship to Proper Time and Proper Length:}
    \begin{itemize}
        \item \textbf{Proper Time ($\Delta \tau$):} In the frame where the particle is stationary, the interval relates to proper time.
        \item \textbf{Proper Length ($\Delta L$):} In the frame where events are simultaneous, the interval relates to proper length.
    \end{itemize}
\end{itemize}
}

\conceptbox{
\textbf{Four-Vectors}
\begin{itemize}
    \item \textbf{Four-Position:} Represents the spacetime coordinates of an event:
    \[
    x^\mu = (ct, \vec{x}) = (ct, x, y, z)
    \]
    where $ct$ is the time component, and $\vec{x} = (x, y, z)$ is the spatial position in 3D space.

    \item \textbf{Four-Velocity:} Describes the rate of change of four-position with respect to proper time $\tau$:
    \[
    u^\mu = \frac{dx^\mu}{d\tau} = \gamma(c, \vec{v})
    \]
    where $\vec{v} = (v_x, v_y, v_z)$ is the 3-velocity and $\gamma = \frac{1}{\sqrt{1 - v^2/c^2}}$.
    \begin{itemize}
        \item The invariant magnitude of four-velocity is $u^\mu u_\mu = c^2$.
    \end{itemize}

    \item \textbf{Four-Momentum:} Represents the energy and momentum of a particle:
    \[
    p^\mu = m_0 u^\mu = (\frac{E}{c}, \vec{p})
    \]
    where $\vec{p} = \gamma m_0 \vec{v}$ is the 3-momentum, and $E = \gamma m_0 c^2$ is the total energy of the particle.
    \begin{itemize}
        \item The invariant magnitude of four-momentum is $p^\mu p_\mu = m_0^2 c^2$, related to the rest energy.
    \end{itemize}

    \item \textbf{Dot Products of Four-Vectors:}
    \begin{itemize}
        \item Inner products between four-vectors are invariant under Lorentz transformations.
        \item Example: For two four-momentum vectors $p^\mu$ and $q^\mu$, the dot product:
        \[
        p^\mu q_\mu = \frac{E_p E_q}{c^2} - \vec{p} \cdot \vec{q}
        \]
        is invariant and depends on the relative energies and momenta in 3D space.
    \end{itemize}

    \item \textbf{Spacetime Diagrams:} Four-vectors can be visualized in spacetime diagrams:
    \begin{itemize}
        \item Time-like vectors point mostly along the time axis.
        \item Space-like vectors point mostly along spatial axes.
        \item Light-like vectors lie on the cone separating time-like and space-like regions.
    \end{itemize}
\end{itemize}
}

\conceptbox{
\textbf{Relativistic Conservation Laws:} 
In an isolated system of particles:
\begin{itemize}
    \item The relativistic total energy (kinetic energy plus rest energy) remains constant.
    \item The total linear momentum remains constant.
\end{itemize}
}

\subsection{Essential Equations}
\eqbox{
\textbf{Time Dilation:} $\Delta t = \gamma \Delta t_0$, where $\gamma = \frac{1}{\sqrt{1 - v^2/c^2}}$.
}

\eqbox{
\textbf{Length Contraction:} $L = L_0 \gamma^{-1} = L_0 \sqrt{1 - v^2/c^2}$.
}

\eqbox{
\textbf{Relativistic Addition of Velocities:} $u' = \displaystyle\frac{u + v}{1 + uv/c^2}$.
}

\eqbox{
\textbf{Lorentz Transformations:}
\begin{itemize}
    \item \textbf{Time:} $t' = \gamma(t - \frac{vx}{c^2})$.
    \item \textbf{Space:} $x' = \gamma(x - vt)$.
\end{itemize}
}

\eqbox{
\textbf{Relativistic Dynamics:}
\begin{itemize}
    \item \textbf{Kinetic Energy:} $KE = (\gamma - 1)m_0c^2$.
    \item \textbf{Total Energy:} $E = \gamma m_0c^2$.
\end{itemize}
}

\eqbox{
\textbf{Energy-Momentum Relation:} $E^2 = (pc)^2 + (m_0c^2)^2$.
}

\eqbox{
\textbf{Four-Position:}
\begin{itemize}
    \item Defined as $x^\mu = (ct, x, y, z)$, where $x^\mu$ is the spacetime coordinate.
    \item Invariant interval: $s^2 = (ct)^2 - x^2 - y^2 - z^2$.
\end{itemize}
}

\eqbox{
\textbf{Four-Velocity:}
\begin{itemize}
    \item Defined as $u^\mu = \frac{dx^\mu}{d\tau}$, where $\tau$ is proper time.
    \item Invariant dot product: $u^\mu u_\mu = c^2$.
\end{itemize}
}

\eqbox{
\textbf{Four-Momentum:}
\begin{itemize}
    \item Defined as $p^\mu = m_0 u^\mu$, where $m_0$ is the rest mass.
    \item Invariant: $p^\mu p_\mu = -m_0^2c^2$.
\end{itemize}
}


\section{Early Quantum Theory and Quantum Mechanics}

\subsection{Key Concepts}
\conceptbox{
\begin{itemize}
    \item Blackbody radiation: Explained using Planck's quantum hypothesis $E = nhf$.
    \item Photoelectric effect: Energy of emitted electrons depends on frequency, not intensity.
    \item Wave-particle duality: Matter exhibits both particle and wave properties.
    \item Schr\"odinger equation: Governs quantum mechanical behavior of particles.
\end{itemize}
}

\subsection{Essential Equations}
\eqbox{
\begin{itemize}
    \item Planck's law: $E = hf$
    \item Photoelectric equation: $KE_{\text{max}} = hf - \phi$
    \item De Broglie wavelength: $\lambda = \frac{h}{p}$
    \item Schrödinger equation (time-independent): 
    \[
    -\frac{\hbar^2}{2m}\frac{d^2\psi}{dx^2} + V\psi = E\psi
    \]
    \item 3D Schrödinger equation (spherical coordinates):
    \[
    -\frac{\hbar^2}{2m} \nabla^2 \psi + V(r)\psi = E\psi
    \]
    where 
    \[
    \nabla^2 = \frac{1}{r^2} \frac{\partial}{\partial r} \left( r^2 \frac{\partial}{\partial r} \right) 
    + \frac{1}{r^2 \sin\theta} \frac{\partial}{\partial \theta} \left( \sin\theta \frac{\partial}{\partial \theta} \right) 
    + \frac{1}{r^2 \sin^2\theta} \frac{\partial^2}{\partial \phi^2}
    \]
    \item General solution to the wave function:
    \[
    \psi(r, \theta, \phi) = R(r) Y(\theta, \phi)
    \]
    where:
    \begin{itemize}
        \item $R(r)$: Radial part of the wave function, satisfying a differential equation dependent on the potential $V(r)$.
        \item $Y(\theta, \phi)$: Angular part, given by spherical harmonics:
        \[
        Y_l^m(\theta, \phi) = (-1)^m \sqrt{\frac{(2l+1)(l-m)!}{4\pi (l+m)!}} P_l^m(\cos\theta)e^{im\phi}
        \]
        where $P_l^m$ are associated Legendre polynomials, and $l, m$ are quantum numbers.
    \end{itemize}
\end{itemize}
}


\section{Statistical Mechanics}

\subsection{Key Concepts}
\conceptbox{
\begin{itemize}
    \item Microstates and macrostates: Connection between microscopic and macroscopic descriptions.
    \item Boltzmann distribution: $P_i = \frac{e^{-E_i/k_BT}}{Z}$.
    \item Partition function: Summation of states to understand thermodynamic properties, $Z = \sum_i e^{-E_i/k_BT}$.
    \item Occupation numbers: Average number of particles in a state, $n_i = \frac{1}{e^{(E_i-\mu)/k_BT} \pm 1}$ (for fermions and bosons).
    \item Degeneracy: The number of microstates corresponding to a single energy level.
    \item Distinguishable vs. indistinguishable particles: Boltzmann distribution for distinguishable particles; Fermi-Dirac and Bose-Einstein distributions for indistinguishable particles.
\end{itemize}
}

\subsection{Essential Equations}
\eqbox{
\begin{align*}
    &\text{Boltzmann distribution: } P_i = \frac{e^{-E_i/k_BT}}{Z} \\
    &\text{Partition function: } Z = \sum_i g_i e^{-E_i/k_BT} \\
    &\text{Fermi-Dirac distribution: } n_i = \frac{1}{e^{(E_i-\mu)/k_BT} + 1} \\
    &\text{Bose-Einstein distribution: } n_i = \frac{1}{e^{(E_i-\mu)/k_BT} - 1} \\
    &\text{Average energy: } \langle E \rangle = \sum_i P_i E_i = -\frac{\partial \ln Z}{\partial \beta} \\
    &\text{Probability of a state: } P(E) = \frac{g_i e^{-E/k_BT}}{Z}
\end{align*}
}

\subsection{Units and Unit Conversions}
\tipbox{
\begin{itemize}
    \item Energy: $1 \text{ eV} = 1.602 \times 10^{-19} \text{ J}$.
    \item Temperature: Convert between Kelvin and energy using $k_B = 1.38 \times 10^{-23} \text{ J/K}$.
    \item Wavelength and frequency: $E = hf$, $\lambda = \frac{c}{f}$, with $hc = 1240 \text{ eV} \cdot \text{nm}$.
\end{itemize}
}

\section{Quick Reference}

\subsection{Mathematical Tools}
\begin{itemize}
    \tipbox{
    \item Trigonometric Identities:
    \begin{align*}
        &\cos^2x = \frac{1 + \cos(2x)}{2} \\
        &\sin^2x = \frac{1 - \cos(2x)}{2} \\
        &\sin(2x) = 2\sin x\cos x
    \end{align*}
    }
    \tipbox{
    \item Series: 
    \begin{itemize}
        \item Taylor Series Expansion: $\displaystyle\sum_{n = 0}^{\infty} C_n(x-a)^n$ where $ C_n = \displaystyle\frac{f^{(n)}(a)}{n!} $
        \item Binomial Series Expansion: $ (1 \pm x)^r = 1 \pm rx \pm ... $
    \end{itemize}
    }
    \tipbox{
    \item Integration Tips:
    \begin{itemize}
        \item For even functions: $\int_{-a}^a f(x)dx = 2\int_0^a f(x)dx$
        \item For odd functions: $\int_{-a}^a f(x)dx = 0$
        \item Gaussian integrals often appear in wave packets
    \end{itemize}
    }

    \tipbox{
    \item Complex Numbers Review
    \begin{itemize}
        \item Complex conjugate: $z^* = a - bi$ for $z = a + bi$
        \item Modulus: $|z| = \sqrt{a^2 + b^2}$
        \item Euler's formula: $e^{i\theta} = \cos\theta + i\sin\theta$
        \item Argument: $\arg(z) = \tan^{-1}\left(\displaystyle\frac{y}{x}\right)$
    \end{itemize}
    }

    \tipbox{
        \item Second Order Differential Equations
        \begin{itemize}
            \item If ODE has this form: $\displaystyle\frac{d^2 f}{dx^2} = k^2 f$ where k is a constant

            The solution is:
            $$ f(x) = Ae^{kx} + Be^{-kx} $$
        \end{itemize}
    }
\end{itemize}

\subsection{Important Constants}
\tipbox{
\begin{itemize}
    \item Stefan-Boltzmann constant: $\sigma = \stefan$
    \item Planck's constant: $h = \planckconst = 4.136\times10^{-15}\text{ eV}\cdot\text{s}$
    \item Reduced Planck's constant: $\hbar = \redplanck$
    \item Electron mass: $m_e = \electronmass$
    \item Speed of light: $c = \lightspeed$
    \item Bohr radius: $a_0 = \bohrradius$
    \item Compton wavelength of the electron: $\displaystyle\frac{h}{m_ec} = \compton$
\end{itemize}
}

\subsection{Unit Conversions}
\tipbox{
\begin{itemize}
    \item $1 \text{ eV} = 1.602 \times 10^{-19} \text{ J}$
    \item $hc = 1240$ eV $\cdot$ nm
    \item $\hbar c = \displaystyle\frac{1240}{2\pi}$ eV $\cdot$ nm
\end{itemize}
}

\end{document}
