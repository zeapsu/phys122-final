\documentclass{article}
\usepackage[a4paper, total={6in, 8in}]{geometry}
\usepackage{amsmath}
\usepackage{amssymb}
\usepackage{physics}
\usepackage{mathtools}
\usepackage{enumitem}
\usepackage{tcolorbox}
\usepackage{graphicx}
\usepackage{setspace}
\usepackage{siunitx}
\usepackage{tikz}
\usepackage{pgfplots}
\usepackage{float}

\usepackage{hyperref}
\hypersetup{
    colorlinks=true,
    linkcolor=black,
    filecolor=magenta,      
    urlcolor=blue,
    pdftitle={Modern Physics Final Study Guide},
    pdfpagemode=FullScreen,
}

\usepackage{fancyhdr}
\pagestyle{fancy}
\fancyhf{} % clear all header and footer fields
\fancyhead[L]{\leftmark} % put section name on the left
\fancyhead[R]{\thepage} % page number on the right
\renewcommand{\headrulewidth}{0.4pt} % horizontal line under header
\renewcommand{\sectionmark}[1]{\markboth{\small#1}{}}

% To handle the first page differently
\fancypagestyle{plain}{
    \fancyhf{} % clear all header and footer fields
    \renewcommand{\headrulewidth}{0pt} % no line for first page
}

\onehalfspacing

\pgfplotsset{compat=1.18}
\usetikzlibrary{angles,quotes,arrows.meta}

\AtBeginDocument{\RenewCommandCopy\qty\SI}

% Custom commands for consistent formatting
\newcommand{\eqbox}[1]{\begin{tcolorbox}[colback=gray!10] #1 \end{tcolorbox}}
\newcommand{\conceptbox}[1]{\begin{tcolorbox}[colback=blue!10] #1 \end{tcolorbox}}
\newcommand{\tipbox}[1]{\begin{tcolorbox}[colback=green!10] #1 \end{tcolorbox}}
\newcommand{\warningbox}[1]{\begin{tcolorbox}[colback=red!10] #1 \end{tcolorbox}}

% Constants
\newcommand{\planckconst}{6.626\times 10^{-34}\,\text{J}\cdot\text{s}}
\newcommand{\redplanck}{1.055\times 10^{-34}\,\text{J}\cdot\text{s}}
\newcommand{\electronmass}{9.11\times 10^{-31}\,\text{kg}}
\newcommand{\lightspeed}{3.0\times 10^{8}\,\text{m}/\text{s}}
\newcommand{\bohrradius}{0.0529\times 10^{-9}\,\text{m}}
\newcommand{\stefan}{5.67\times10^{-8}\text{ W/m}^2\cdot K^4}
\newcommand{\compton}{0.002426\text{ nm}}

\title{Modern Physics Final Study Guide}
\author{PHYS-122}
\date{Fall 2024}

\begin{document}
\maketitle

\newpage

\tableofcontents

\newpage

\section{Special Relativity}

\subsection{Postulates of Special Relativity}
\conceptbox{
\textbf{Postulates:}
\begin{itemize}
    \item The laws of physics are the same in all inertial frames.
    \item The speed of light is constant in a vacuum for all observers, regardless of their motion relative to the source.
\end{itemize}
}

\subsection{Relativity of Simultaneity}
\conceptbox{
\textbf{Relativity of Simultaneity:}
Two events that appear simultaneous to one observer may not appear simultaneous to another observer moving relative to the first. This is a direct consequence of the constancy of the speed of light and the Lorentz transformations.
}

\subsection{Lorentz Transformations}

\conceptbox{
\textbf{Lorentz Transformations:}
The Lorentz transformations relate spacetime coordinates of an event in two inertial reference frames moving at a constant velocity relative to each other. These transformations preserve the invariant interval (\( \Delta s^2 \)), a cornerstone of special relativity.
}

\conceptbox{
\textbf{Transformations:}
\begin{itemize}
    \item For a frame \( S' \) moving with velocity \( v \) along the \( x \)-axis relative to \( S \):
    \eqbox{
    \begin{itemize}
        \item Time: \( t' = \gamma \left( t - \frac{vx}{c^2} \right) \)
        \item Space (\( x \)-direction): \( x' = \gamma \left( x - vt \right) \)
        \item Space (\( y \)- and \( z \)-directions): \( y' = y, \quad z' = z \)
    \end{itemize}
    }
    \item The Lorentz factor is \( \gamma = \frac{1}{\sqrt{1 - v^2/c^2}} \).
\end{itemize}
}

\conceptbox{
\textbf{Lorentz Transformation Matrix:}
The Lorentz transformation can be expressed using matrix multiplication:
\eqbox{
\[
x' = \Lambda x,
\]
}
where \( x = \begin{bmatrix} ct \\ x \\ y \\ z \end{bmatrix} \) and \( x' \) are the spacetime coordinates in \( S \) and \( S' \), respectively. For a boost along the \( x \)-axis, the Lorentz transformation matrix \( \Lambda \) is:
\eqbox{
\[
\Lambda =
\begin{bmatrix}
\gamma & -\gamma \beta & 0 & 0 \\
-\gamma \beta & \gamma & 0 & 0 \\
0 & 0 & 1 & 0 \\
0 & 0 & 0 & 1
\end{bmatrix},
\]
}
where \( \beta = v/c \).
}

\conceptbox{
\textbf{Key Properties:}
\begin{itemize}
    \item The inverse transformation is obtained by negating \( v \), flipping the sign of \( \beta \):
    \eqbox{
    \[
    \Lambda^{-1} = \begin{bmatrix}
    \gamma & \gamma \beta & 0 & 0 \\
    \gamma \beta & \gamma & 0 & 0 \\
    0 & 0 & 1 & 0 \\
    0 & 0 & 0 & 1
    \end{bmatrix}.
    \]
    }
    \item The Lorentz transformations preserve the invariant interval:
    \eqbox{
    \[
    \Delta s^2 = x \cdot x = x'\cdot x'.
    \]
    }
\end{itemize}
}

\conceptbox{
\textbf{Applications:}
\begin{itemize}
    \item Lorentz transformations explain phenomena like time dilation, length contraction, and the relativity of simultaneity.
    \item They form the basis for defining and transforming four-vectors, such as four-momentum and four-velocity.
\end{itemize}
}


\subsection{The Relativistic Invariant Interval}
\conceptbox{
\textbf{Invariant Interval:}
The relativistic invariant interval is a spacetime quantity that remains unchanged under Lorentz transformations.
}
\eqbox{
\textbf{Invariant Interval:}
\[
\Delta s^2 = (c\Delta t)^2 - (\Delta x)^2 - (\Delta y)^2 - (\Delta z)^2
\]}
\conceptbox{
\textbf{Classifications of the Invariant Interval:}
\begin{itemize}
    \item \textbf{Time-like:} \( \Delta s^2 > 0 \), related to proper time:
    \eqbox{
    \[
    \Delta s^2 = c^2 \Delta \tau^2
    \]
    }
    \item \textbf{Space-like:} \( \Delta s^2 < 0 \), related to proper length:
    \eqbox{
    \[
    -\Delta s^2 = L^2
    \]
    }
    \item \textbf{Light-like:} \( \Delta s^2 = 0 \), events are connected by light:
    \eqbox{
    \[
    c^2\Delta t^2 = (\Delta x)^2 + (\Delta y)^2 + (\Delta z)^2
    \]
    }
\end{itemize}
}

\subsection{Relativistic Velocity Addition}
\conceptbox{
\textbf{Relativistic Velocity Addition:}
When combining velocities in different inertial frames, the relativistic formula must be used. Typically, these problems involve 3 objects and the following reference frames and velocities are defined:
\begin{itemize}
    \item \( S \): The stationary reference frame.
    \item \( S' \): The moving reference frame.
    \item \( u \): The velocity of an object relative to \( S \).
    \item \( v \): The velocity of \( S' \) relative to \( S \).
    \item \( u' \): The velocity of the object relative to \( S' \).
\end{itemize}
\eqbox{
\textbf{Relativistic Velocity Addition Formula:}
\[
u' = \frac{u + v}{1 + \frac{uv}{c^2}}
\]
Here, \( u' \) is the velocity of the object as measured in \( S' \), given the velocity \( u \) of the object relative to \( S \) and the velocity \( v \) of \( S' \) relative to \( S \).
}
}


\subsection{Four-Vectors in Special Relativity}
\conceptbox{
\textbf{Four-Vectors:}
Physical quantities in relativity are represented as four-vectors that transform under Lorentz transformations.
}
\eqbox{
\textbf{General Form of a Four-Vector:}
\[
a^\mu = (a^0, \vec{a}) = (a^0, a^1, a^2, a^3), \quad a^0 = c\Delta t
\]
}
\conceptbox{
\textbf{Dot Product of Four-Vectors:}
The invariant dot product between two four-vectors \( a^\mu \) and \( b^\mu \) is:
\eqbox{
\[
a \cdot b = a^0 b^0 - \vec{a} \cdot \vec{b}
\]
}
This quantity is Lorentz invariant (unchanged under Lorentz transformations).
}

\subsection{Examples of Four-Vectors}
\conceptbox{
\textbf{Common Four-Vectors:}
\begin{itemize}
    \item \textbf{Four-Position:} 
    \eqbox{
    \[
    x^\mu = (ct, x, y, z)
    \]
    }
    \item \textbf{Four-Velocity:} 
    \eqbox{
    \[
    u^\mu = \frac{dx^\mu}{d\tau} = \gamma (c, \vec{v}), \quad \gamma = \frac{1}{\sqrt{1 - v^2/c^2}}
    \]
    }
    \item \textbf{Four-Momentum:} 
    \eqbox{
    \[
    p^\mu = m u^\mu = \gamma m (c, \vec{v})
    \]
    }
    \eqbox{
    For massless particles, like photons:
    \[
    p \cdot p = 0, \quad E = |\vec{p}|c
    \]
    }
\end{itemize}
}

\subsection{Relativistic Conservation Laws}

\conceptbox{
\textbf{Relativistic Conservation Laws:}
The principles of conservation of energy and momentum extend to special relativity, incorporating the total energy (rest energy + kinetic energy) and relativistic momentum.
}

\conceptbox{
    \textbf{Relativistic Total Energy:}
    \begin{itemize}
        \item The total energy \( E \) includes both the rest energy and the relativistic kinetic energy:
        \eqbox{
        \[
        E = \gamma m_0 c^2.
        \]
        }
        \item The rest energy is given by \( E_0 = m_0 c^2 \), where \( m_0 \) is the rest mass.
        \item In any physical process, the total energy is conserved.
    \end{itemize}
}
\conceptbox{
    \textbf{Relativistic Momentum:}
    \begin{itemize}
        \item The relativistic momentum is defined as:
        \eqbox{
        \[
        \vec{p} = \gamma m_0 \vec{v},
        \]
        }
        where \( \gamma = \frac{1}{\sqrt{1 - v^2/c^2}} \).
        \item Both the magnitude and direction of total momentum in an isolated system are conserved.
    \end{itemize}
}

\conceptbox{
    \textbf{Four-Momentum Conservation:}
    \begin{itemize}
        \item Energy and momentum form components of the four-momentum vector:
        \eqbox{
        \[
        p^\mu = \left( \frac{E}{c}, \vec{p} \right).
        \]
        }
        \item In all inertial frames, the conservation of four-momentum holds:
        \eqbox{
        \[
        \sum p^\mu_{\text{initial}} = \sum p^\mu_{\text{final}}.
        \]
    }
    \end{itemize}
}

\conceptbox{
    \textbf{Relativistic Collisions:}
    \begin{itemize}
        \item Total energy and total momentum are conserved in collisions, including elastic and inelastic cases.
        \item Unlike classical mechanics, the concept of conserved "mass" is not generally applicable. For example:
        \begin{itemize}
            \item In pair annihilation, rest mass converts to energy.
            \item In particle creation, energy converts into rest mass.
        \end{itemize}
    \end{itemize}
}

\eqbox{
\textbf{Energy-Momentum Relation:}
\[
E^2 = (pc)^2 + (m_0c^2)^2
\]
This relation connects the total energy \( E \), the momentum \( p \), and the rest mass \( m_0 \) of a particle.
}

\eqbox{
\textbf{Invariant Four-Momentum Magnitude:}
\[
p^\mu \cdot p^\mu = m_0^2 c^2
\]
This invariant magnitude holds in all inertial frames, ensuring the consistency of relativistic conservation laws.
}
\newpage
\section{Quantum Mechanics}

\subsection{Quantum Theory of Light}

\subsubsection{Blackbody Radiation and Planck Hypothesis}
\conceptbox{
\textbf{Blackbody Radiation:} 
Classical physics predicted the "ultraviolet catastrophe," where energy radiated at high frequencies diverged. Planck resolved this by quantizing energy:
\eqbox{
\[
E = nhf, \quad n = 1, 2, 3, \dots
\]}
\textbf{Planck's Law:}
\eqbox{
\[
I(\nu, T) = \frac{8\pi h\nu^3}{c^3}\frac{1}{e^{h\nu/k_BT} - 1},
\]
}
where \( I(\nu, T) \) is the intensity of radiation at frequency \( \nu \).
}

\subsubsection{Photoelectric Effect}
\conceptbox{
\textbf{Photoelectric Effect:}
Electrons are emitted from a material when light shines on it, with the energy of the electrons depending on the frequency of light.
\eqbox{
\[
hf = \phi + KE_{max}, \quad hf_c = \phi, \quad KE_{max} = eV_s
\]}
\textbf{Threshold Frequency:} No electrons are emitted if \( f < f_c \).
}

\subsubsection{Compton Scattering}
\conceptbox{
\textbf{Compton Scattering:}
The scattering of X-rays by electrons demonstrates particle-like behavior of light. The wavelength shift is given by:
\eqbox{
\[
\lambda' - \lambda = \frac{h}{m_ec}(1 - \cos\theta).
\]}
\textbf{Energy Relation:}
\eqbox{
\[
\frac{1}{E'} - \frac{1}{E} = \frac{1 - \cos\theta}{m_ec^2}.
\]
}
}
\newpage
\subsubsection{Wave-Particle Duality}

\conceptbox{
\textbf{Wave-Particle Duality:}
Light and matter exhibit dual behavior, displaying characteristics of both waves and particles depending on the experimental conditions.
}
\conceptbox{
\textbf{Key Features:}
\begin{itemize}
    \item Travels as a wave, interacting with itself, and demonstrates phenomena such as interference and diffraction.
    \item Interacts as a particle, transferring discrete packets of energy, known as photons or quanta.
    \item Confirmed by experiments:
    \begin{itemize}
        \item \textbf{Double-Slit Experiment:} Demonstrates interference patterns for light and electrons, showing wave-like behavior.
        \item \textbf{Photoelectric Effect:} Demonstrates the particle nature of light, as photons eject electrons from a material.
        \item \textbf{Blackbody Radiation:} Explained by Planck's quantization hypothesis, resolving the ultraviolet catastrophe.
        \item \textbf{Compton Scattering:} Demonstrates photon momentum through scattering with electrons.
    \end{itemize}
\end{itemize}
\eqbox{
\textbf{De Broglie Wavelength:} \( \lambda = \displaystyle\frac{h}{p} = \displaystyle\frac{h}{mv} \)
}

\eqbox{
\textbf{Wave Number:} \( k = \displaystyle\frac{2\pi}{\lambda} \)
}
}

\subsubsection{Heisenberg Uncertainty Principle}

\conceptbox{
\textbf{Heisenberg Uncertainty Principle:}
A fundamental principle of quantum mechanics stating that certain pairs of physical observables cannot be simultaneously measured with arbitrary precision.
}

\conceptbox{
\textbf{Key Relations:}
\begin{itemize}
    \item \textbf{Position-Momentum Uncertainty:}
    \eqbox{
    \[
    \Delta x \Delta p \geq \frac{\hbar}{2},
    \]
    }
    where \( \Delta x \) and \( \Delta p \) are the standard deviations of position and momentum, respectively.
    \item \textbf{Energy-Time Uncertainty:}
    \eqbox{
    \[
    \Delta E \Delta t \geq \frac{\hbar}{2},
    \]
    }
    where \( \Delta E \) and \( \Delta t \) are the uncertainties in energy and time, respectively.
\end{itemize}
}

\conceptbox{
\textbf{Variance-Squared Average Relationship:}
The uncertainty in a measurable quantity \( A \) is defined as:
\eqbox{
\[
\Delta A = \sqrt{\langle A^2 \rangle - \langle A \rangle^2},
\]
}
where \( \langle A \rangle \) is the expectation value, and \( \langle A^2 \rangle \) is the expectation value of \( A^2 \).
}

\conceptbox{
\textbf{Implications:}
\begin{itemize}
    \item Establishes the probabilistic nature of quantum mechanics, replacing deterministic classical physics.
    \item Demonstrates the impossibility of assigning definite trajectories to particles as in classical mechanics.
\end{itemize}
}

\subsubsection{Group and Phase Velocities}
\conceptbox{
\textbf{Group and Phase Velocities:}
Velocities associated with wave propagation.
\begin{itemize}
    \item Group velocity:
    \eqbox{
    \[
    v_g = \frac{d\omega}{dk},
    \]
    }
    representing the velocity of the wave packet \textbf{and the velocity of the particle}
    \item Phase velocity:
    \eqbox{
    \[
    v_p = \frac{\omega}{k},
    \]
    }
    representing the velocity of individual wave crests.
\end{itemize}
}


\subsection{Schrödinger Equation}

\subsubsection{Overview}
\conceptbox{
\textbf{Schrödinger Equation:} Governs the quantum mechanical behavior of particles. It describes how the wave function of a particle evolves under the influence of potential energy.

\textbf{General Forms:}
\begin{itemize}
    \item \textbf{1D Time-Independent Schrödinger Equation:}
    \eqbox{
    \[
    -\frac{\hbar^2}{2m} \frac{d^2\psi(x)}{dx^2} + V(x)\psi(x) = E\psi(x).
    \]}
    \item \textbf{3D Schrödinger Equation (spherical coordinates):}
    \eqbox{
    \[
    -\frac{\hbar^2}{2m} \nabla^2 \psi(r, \theta, \phi) + V(r)\psi(r, \theta, \phi) = E\psi(r, \theta, \phi).
    \]}
    The solution is separable:
    \eqbox{
    \[
    \psi(r, \theta, \phi) = R(r)Y(\theta, \phi),
    \]}
    where \( Y(\theta, \phi) \) are spherical harmonics.
\end{itemize}
}

\subsubsection{Different Potentials}
\conceptbox{
\textbf{Free Particle:}
\begin{itemize}
    \item For \( V(x) = 0 \), solutions are plane waves:
    \eqbox{
    \[
    \psi(x) = Ae^{ikx} + Be^{-ikx},
    \]}
    where \( k = \frac{p}{\hbar} \), \( \omega = \frac{E}{\hbar} \), and \(p = \hbar k = \sqrt{2mE}\).
    \item \(e^{ikx}\) is right moving plane wave and \(e^{-ikx}\) is left moving plane wave.
    \item For full time dependent, \(\Psi(x,t) = \psi(x)e^{-i\omega t}\)
    \item Wave function is not normalizable!
\end{itemize}
}

\conceptbox{
\textbf{Particle in a Box/Infinite Square Well Potential:}
\begin{itemize}
    \item For \( V(x) = 0 \) inside the box and \( V(x) = \infty \) outside:
    \begin{itemize}
        \item General Solution 1D:
            \eqbox{
                \[
                    \psi_n(x) = A\sin\left(\frac{n\pi x}{L}\right)
                \]
            }
        \item General Solution 3D:
            \eqbox{
                \[
                \psi_{n_1, n_2, n_3} = A\sin(\frac{n_1 \pi x}{L})\sin(\frac{n_2 \pi y}{L})\sin(\frac{n_3 \pi z}{L})
                , \quad E_n = \frac{\hbar ^2 \pi^2}{2mL^2}(n_1^2 + n_2^2 + n_3^2)
                \]
            }
        \item Particular Solution 1D:
            \eqbox{
            \[
            \psi_n(x) = \sqrt{\frac{2}{L}} \sin\left(\frac{n\pi x}{L}\right), \quad E_n = \frac{n^2\pi^2\hbar^2}{2mL^2} = \frac{n^2 h^2}{8mL^2}.
            \]}
    \end{itemize}
\end{itemize}
}


\conceptbox{
\textbf{Step Potential:}
\[
V(x) = 
\begin{cases} 
0, & x < 0 \\ 
U_0, & x \geq 0 
\end{cases}
\]
The particle's behavior depends on whether its energy \( E \) is greater than or less than \( U_0 \).
}

\conceptbox{
\subsubsection*{Case 1: \( E > U_0 \)}

\textbf{Wave Function Solutions:}
\begin{itemize}
    \item \( x < 0 \): \( \psi(x) = Ae^{ik_0x} + Be^{-ik_0x}, \quad k_0 = \sqrt{\frac{2mE}{\hbar^2}} \)
    \item \( x \geq 0 \): \( \psi(x) = Ce^{ik_1x}, \quad k_1 = \sqrt{\frac{2m(E - U_0)}{\hbar^2}} \)
\end{itemize}

\textbf{Reflection and Transmission Coefficients:}
\[
R = \left( \frac{k_0 - k_1}{k_0 + k_1} \right)^2, \quad T = \frac{4k_0k_1}{(k_0 + k_1)^2}, \quad R + T = 1
\]
}


\conceptbox{
\subsubsection*{Case 2: \( E < U_0 \)}
\textbf{Wave Function Solutions:}
\begin{itemize}
    \item \( x < 0 \): \( \psi(x) = Ae^{ik_0x} + Be^{-ik_0x}, \quad k_0 = \sqrt{\frac{2mE}{\hbar^2}} \)
    \item \( x \geq 0 \): \( \psi(x) = De^{-\kappa x}, \quad \kappa = \sqrt{\frac{2m(U_0 - E)}{\hbar^2}} \)
\end{itemize}

\textbf{Tunneling Insight:}
The particle cannot propagate in the \( x \geq 0 \) region but has a finite probability of being found in the barrier due to tunneling.
}

\conceptbox{
\textbf{Key Insights:}
\begin{itemize}
    \item For \( E > U_0 \), the particle has probabilities of transmission and reflection at the step.
    \item For \( E < U_0 \), the particle exhibits tunneling with an exponentially decaying wave function in the barrier.
\end{itemize}
}


\subsection{Hydrogen Atom}

\subsubsection{Wave Functions}
\conceptbox{
\textbf{Wave Functions:}
\eqbox{
\[
\psi(r, \theta, \phi) = R_{nl}(r)Y_l^m(\theta, \phi),
\]}
where \( R_{nl}(r) \) is the radial wave function, and \( Y_l^m(\theta, \phi) \) are spherical harmonics.

\textbf{Radial Probability Distribution:}
The probability density \( P(r) \) is given by:
\eqbox{
\[
P(r) = r^2|R_{nl}(r)|^2.
\]}
}

\subsubsection{Quantum Numbers}
\conceptbox{
\textbf{Quantum Numbers:}
\begin{itemize}
    \item \textbf{Principal quantum number (\( n \)):} Determines energy level (\( n = 1, 2, 3, \dots \)).
    \begin{itemize}
        \item Tells us the \textbf{energy} of the Hydrogen atom
        \eqbox{
            \[
                E(\psi_{n, \ell, m}) = -\frac{13.6\text{ eV}}{n^2}
            \]
        }
    \end{itemize}
    \item \textbf{Orbital angular momentum quantum number (\( \ell \)):} Determines the shape of the orbital (\( \ell = 0, 1, \dots, n-1 \)).
    \begin{itemize}
        \item Tells us the \textbf{total angular momentum} of the Hydrogen atom
        \eqbox{
            \[
                L(\psi_{n, \ell, m}) = \hbar \sqrt{\ell(\ell + 1)}
            \]
        }
    \end{itemize}
    \item \textbf{Magnetic quantum number (\( m \)):} Determines the orientation (\( m = -\ell, -\ell+1, \dots, \ell \)).
    \begin{itemize}
        \item Tells us the \textbf{z-component of angular momentum} of the Hydrogen atom
        \eqbox{
            \[
                L_z(\psi_{n, \ell, m}) = \hbar m
            \]
        }
    \end{itemize}
    \item \textbf{Spin quantum number (\( m_s \)):} Intrinsic angular momentum (\( m_s = \pm \frac{1}{2} \)).
    \begin{itemize}
        \item Has to be rotated \textbf{720 degrees} for it to return to original state
        \item Fermions have half integer spin (electrons, muons, protons, neutrons, etc.)
        \item Bosons have integer spin (photons, W and Z bosons, Higgs boson, etc.)
        \item Stern-Gerlach Experiment
    \end{itemize}
\end{itemize}
}

\subsubsection{Multi-Electron Atoms and the Pauli Exclusion Principle}

\conceptbox{
\textbf{Multi-Electron Atoms:}
In atoms with more than one electron, the arrangement of electrons is determined by their energies and quantum states:
\begin{itemize}
    \item Electrons occupy orbitals starting from the lowest energy levels, but no more than two electrons can share the same orbital.
    \item Inner electrons shield outer electrons from the full nuclear charge, affecting orbital energies.
    \item Example: In Potassium (\( Z = 19 \)), the electron configuration is \( 1s^2 2s^2 2p^6 3s^2 3p^6 4s^1 \). The last electron occupies the \( 4s \) orbital instead of \( 3d \) due to shielding.
\end{itemize}
}

\conceptbox{
\textbf{Pauli Exclusion Principle:}
No two electrons in an atom can occupy the same quantum state. This means:
\begin{itemize}
    \item No two electrons can have identical quantum numbers (\( n, l, m, m_s \)).
    \item This principle determines the filling of electron orbitals and gives rise to electron configurations.
\end{itemize}
}

\conceptbox{
\textbf{Electron Configuration:}
The distribution of electrons in orbitals follows these rules:
\begin{itemize}
    \item Electrons are added one at a time to the lowest energy subshell available.
    \item The periodic table reflects the outer electron configurations, which dictate chemical properties.
\end{itemize}
}

\section{Statistical Mechanics}

\subsection{Key Concepts}
\conceptbox{
\begin{itemize}
    \item Microstates and macrostates: Connection between microscopic and macroscopic descriptions.
    \item Boltzmann distribution: $P_i = \frac{e^{-E_i/k_BT}}{Z}$.
    \item Partition function: Summation of states to understand thermodynamic properties, $Z = \sum_i e^{-E_i/k_BT}$.
    \item Occupation numbers: Average number of particles in a state, $n_i = \frac{1}{e^{(E_i-\mu)/k_BT} \pm 1}$ (for fermions and bosons).
    \item Degeneracy: The number of microstates corresponding to a single energy level.
    \item Distinguishable vs. indistinguishable particles: Boltzmann distribution for distinguishable particles; Fermi-Dirac and Bose-Einstein distributions for indistinguishable particles.
\end{itemize}
}

\subsection{Essential Equations}
\eqbox{
\begin{align*}
    &\text{Boltzmann distribution: } P_i = \frac{e^{-E_i/k_BT}}{Z} \\
    &\text{Partition function: } Z = \sum_i g_i e^{-E_i/k_BT} \\
    &\text{Fermi-Dirac distribution: } n_i = \frac{1}{e^{(E_i-\mu)/k_BT} + 1} \\
    &\text{Bose-Einstein distribution: } n_i = \frac{1}{e^{(E_i-\mu)/k_BT} - 1} \\
    &\text{Average energy: } \langle E \rangle = \sum_i P_i E_i = -\frac{\partial \ln Z}{\partial \beta} \\
    &\text{Probability of a state: } P(E) = \frac{g_i e^{-E/k_BT}}{Z}
\end{align*}
}

\subsection{Units and Unit Conversions}
\tipbox{
\begin{itemize}
    \item Energy: $1 \text{ eV} = 1.602 \times 10^{-19} \text{ J}$.
    \item Temperature: Convert between Kelvin and energy using $k_B = 1.38 \times 10^{-23} \text{ J/K}$.
    \item Wavelength and frequency: $E = hf$, $\lambda = \frac{c}{f}$, with $hc = 1240 \text{ eV} \cdot \text{nm}$.
\end{itemize}
}

\section{Quick Reference}

\subsection{Mathematical Tools}
\begin{itemize}
    \tipbox{
    \item Trigonometric Identities:
    \begin{align*}
        &\cos^2x = \frac{1 + \cos(2x)}{2} \\
        &\sin^2x = \frac{1 - \cos(2x)}{2} \\
        &\sin(2x) = 2\sin x\cos x
    \end{align*}
    }
    \tipbox{
    \item Series: 
    \begin{itemize}
        \item Taylor Series Expansion: $\displaystyle\sum_{n = 0}^{\infty} C_n(x-a)^n$ where $ C_n = \displaystyle\frac{f^{(n)}(a)}{n!} $
        \item Binomial Series Expansion: $ (1 \pm x)^r = 1 \pm rx \pm ... $
    \end{itemize}
    }
    \tipbox{
    \item Integration Tips:
    \begin{itemize}
        \item For even functions: $\int_{-a}^a f(x)dx = 2\int_0^a f(x)dx$
        \item For odd functions: $\int_{-a}^a f(x)dx = 0$
        \item Gaussian integrals often appear in wave packets
    \end{itemize}
    }

    \tipbox{
    \item Complex Numbers Review
    \begin{itemize}
        \item Complex conjugate: $z^* = a - bi$ for $z = a + bi$
        \item Modulus: $|z| = \sqrt{a^2 + b^2}$
        \item Euler's formula: $e^{i\theta} = \cos\theta + i\sin\theta$
        \item Argument: $\arg(z) = \tan^{-1}\left(\displaystyle\frac{y}{x}\right)$
    \end{itemize}
    }

    \tipbox{
        \item Second Order Differential Equations
        \begin{itemize}
            \item If ODE has this form: $\displaystyle\frac{d^2 f}{dx^2} = k^2 f$ where k is a constant

            The solution is:
            $$ f(x) = Ae^{kx} + Be^{-kx} $$
        \end{itemize}
    }
\end{itemize}

\subsection{Important Constants}
\tipbox{
\begin{itemize}
    \item Stefan-Boltzmann constant: $\sigma = \stefan$
    \item Planck's constant: $h = \planckconst = 4.136\times10^{-15}\text{ eV}\cdot\text{s}$
    \item Reduced Planck's constant: $\hbar = \redplanck$
    \item Electron mass: $m_e = \electronmass$
    \item Speed of light: $c = \lightspeed$
    \item Bohr radius: $a_0 = \bohrradius$
    \item Compton wavelength of the electron: $\displaystyle\frac{h}{m_ec} = \compton$
\end{itemize}
}

\subsection{Unit Conversions}
\tipbox{
\begin{itemize}
    \item $1 \text{ eV} = 1.602 \times 10^{-19} \text{ J}$
    \item $hc = 1240$ eV $\cdot$ nm
    \item $\hbar c = \displaystyle\frac{1240}{2\pi}$ eV $\cdot$ nm
\end{itemize}
}

\end{document}
