\documentclass{article}
\usepackage[a4paper, total={6in, 8in}]{geometry}
\usepackage{amsmath}
\usepackage{amssymb}
\usepackage{physics}
\usepackage{mathtools}
\usepackage{enumitem}
\usepackage{tcolorbox}
\usepackage{graphicx}
\usepackage{setspace}
\usepackage{siunitx}
\usepackage{tikz}
\usepackage{pgfplots}
\usepackage{float}
\usepackage{multicol}

\usepackage{hyperref}
\hypersetup{
    colorlinks=true,
    linkcolor=black,
    filecolor=magenta,      
    urlcolor=blue,
    pdftitle={Modern Physics Final Study Guide},
    pdfpagemode=FullScreen,
}

\usepackage{fancyhdr}
\pagestyle{fancy}
\fancyhf{} % clear all header and footer fields
\fancyhead[L]{\leftmark} % put section name on the left
\fancyhead[R]{\thepage} % page number on the right
\renewcommand{\headrulewidth}{0.4pt} % horizontal line under header
\renewcommand{\sectionmark}[1]{\markboth{\small#1}{}}

% To handle the first page differently
\fancypagestyle{plain}{
    \fancyhf{} % clear all header and footer fields
    \renewcommand{\headrulewidth}{0pt} % no line for first page
}

\onehalfspacing

\pgfplotsset{compat=1.18}
\usetikzlibrary{angles,quotes,arrows.meta}

\AtBeginDocument{\RenewCommandCopy\qty\SI}

% Custom commands for consistent formatting
\newcommand{\eqbox}[1]{\begin{tcolorbox}[colback=gray!10] #1 \end{tcolorbox}}
\newcommand{\conceptbox}[1]{\begin{tcolorbox}[colback=blue!10] #1 \end{tcolorbox}}
\newcommand{\tipbox}[1]{\begin{tcolorbox}[colback=green!10] #1 \end{tcolorbox}}
\newcommand{\warningbox}[1]{\begin{tcolorbox}[colback=red!10] #1 \end{tcolorbox}}

% Constants
\newcommand{\planckconst}{6.626\times 10^{-34}\,\text{J}\cdot\text{s}}
\newcommand{\redplanck}{1.055\times 10^{-34}\,\text{J}\cdot\text{s}}
\newcommand{\electronmass}{9.11\times 10^{-31}\,\text{kg}}
\newcommand{\lightspeed}{3.0\times 10^{8}\,\text{m}/\text{s}}
\newcommand{\bohrradius}{0.0529\times 10^{-9}\,\text{m}}
\newcommand{\stefan}{5.67\times10^{-8}\text{ W/m}^2\cdot K^4}
\newcommand{\compton}{0.002426\text{ nm}}

\title{Modern Physics Final Study Guide}
\author{PHYS-122}
\date{Fall 2024}

\begin{document}
\maketitle

\newpage

\tableofcontents

\newpage

\section{Special Relativity}

\subsection{Postulates of Special Relativity}
\conceptbox{
\textbf{Postulates:}
\begin{itemize}
    \item The laws of physics are the same in all inertial frames.
    \item The speed of light is constant in a vacuum for all observers, regardless of their motion relative to the source.
\end{itemize}
}

\subsection{Relativity of Simultaneity}
\conceptbox{
\textbf{Relativity of Simultaneity:}
Two events that appear simultaneous to one observer may not appear simultaneous to another observer moving relative to the first. This is a direct consequence of the constancy of the speed of light and the Lorentz transformations.
}

\subsection{Galilean Relativity and Transformations}

\subsubsection{Galilean Relativity}
\conceptbox{
\textbf{Galilean Relativity:}
The principle of Galilean relativity states that the laws of mechanics are the same in all inertial frames of reference. No experiment conducted within a closed system can determine whether the system is at rest or moving with constant velocity.

\textbf{Key Assumptions:}
\begin{itemize}
    \item Time is absolute: All observers measure the same time regardless of their motion.
    \item Space is absolute: The distance between two points is the same in all inertial frames.
    \item Velocities add linearly (classical velocity addition).
\end{itemize}
}

\subsubsection{Galilean Transformations}
\conceptbox{
The Galilean transformations describe how coordinates change between two inertial reference frames \( S \) and \( S' \), where \( S' \) moves with a constant velocity \( v \) relative to \( S \) along the \( x \)-axis.

\textbf{Transformations:}
\eqbox{
\[
    x' = x - vt, \quad y' = y, \quad z' = z, \quad t' = t.
\]
}
Here:
\begin{itemize}
    \item \( x, y, z, t \): Position and time in the stationary frame \( S \).
    \item \( x', y', z', t' \): Position and time in the moving frame \( S' \).
    \item \( v \): Relative velocity between \( S \) and \( S' \).
\end{itemize}
}

\conceptbox{
\textbf{Classical Velocity Addition:}
The velocity of an object as observed in \( S' \) is related to its velocity in \( S \) by:
\eqbox{
\[
    v' = v_{\text{object}} - v_{\text{frame}}.
\]
}
This linear velocity addition assumes that velocities are simply additive without any limitations (e.g., no speed limit like the speed of light).
}

\subsubsection{Failures of Galilean Relativity}
\conceptbox{
\textbf{Key Failings of Galilean Relativity:}
Galilean relativity is inconsistent with observations involving light and electromagnetism, leading to its replacement by Einstein's Special Relativity. The main issues include:

\begin{enumerate}
    \item \textbf{Failure to Explain the Constancy of Light Speed:}
    \begin{itemize}
        \item According to Maxwell's equations, the speed of light in a vacuum (\( c \)) is constant and does not depend on the motion of the observer or the source.
        \item The Galilean velocity addition formula predicts that the speed of light would vary depending on the relative motion of the observer, which contradicts experimental evidence (e.g., the Michelson-Morley experiment).
    \end{itemize}
    
    \item \textbf{Absolute Time and Space Assumptions:}
    \begin{itemize}
        \item Galilean relativity assumes that time flows uniformly for all observers (\( t = t' \)) and that space is the same in all frames.
        \item Special Relativity shows that time and space are interconnected (spacetime), and both are affected by the observer's relative motion.
    \end{itemize}

    \item \textbf{No Concept of Lorentz Invariance:}
    \begin{itemize}
        \item The Galilean transformations preserve distances and times but do not preserve the invariant interval (\( \Delta s^2 \)) observed in Special Relativity.
        \item Lorentz transformations, which replace Galilean transformations, ensure that the speed of light remains constant and that physical laws are the same in all inertial frames.
    \end{itemize}
\end{enumerate}
}

\conceptbox{
\textbf{Conclusion:}
The failures of Galilean relativity arise from its inability to reconcile classical mechanics with the behavior of light and electromagnetism. This led to the development of Special Relativity, where:
\begin{itemize}
    \item Space and time are relative and interdependent.
    \item The speed of light (\( c \)) is constant in all inertial frames.
    \item Lorentz transformations replace Galilean transformations to correctly describe spacetime.
\end{itemize}
}


\subsection{Lorentz Transformations}

\conceptbox{
\textbf{Lorentz Transformations:}
The Lorentz transformations relate spacetime coordinates of an event in two inertial reference frames moving at a constant velocity relative to each other. These transformations preserve the invariant interval (\( \Delta s^2 \)), a cornerstone of special relativity.
}

\conceptbox{
\textbf{Transformations:}
\begin{itemize}
    \item For a frame \( S' \) moving with velocity \( v \) along the \( x \)-axis relative to \( S \):
    \eqbox{
    \begin{itemize}
        \item Time: \( t' = \gamma \left( t - \frac{vx}{c^2} \right) \)
        \item Space (\( x \)-direction): \( x' = \gamma \left( x - vt \right) \)
        \item Space (\( y \)- and \( z \)-directions): \( y' = y, \quad z' = z \)
    \end{itemize}
    }
    \item The Lorentz factor is \( \gamma = \frac{1}{\sqrt{1 - v^2/c^2}} \).
\end{itemize}
}

\conceptbox{
\textbf{Lorentz Transformation Matrix:}
The Lorentz transformation can be expressed using matrix multiplication:
\eqbox{
\[
x' = \Lambda x,
\]
}
where \( x = \begin{bmatrix} ct \\ x \\ y \\ z \end{bmatrix} \) and \( x' \) are the spacetime coordinates in \( S \) and \( S' \), respectively. For a boost along the \( x \)-axis, the Lorentz transformation matrix \( \Lambda \) is:
\eqbox{
\[
\Lambda =
\begin{bmatrix}
\gamma & -\gamma \beta & 0 & 0 \\
-\gamma \beta & \gamma & 0 & 0 \\
0 & 0 & 1 & 0 \\
0 & 0 & 0 & 1
\end{bmatrix},
\]
}
where \( \beta = v/c \).
}

\conceptbox{
\textbf{Key Properties:}
\begin{itemize}
    \item The inverse transformation is obtained by negating \( v \), flipping the sign of \( \beta \):
    \eqbox{
    \[
    \Lambda^{-1} = \begin{bmatrix}
    \gamma & \gamma \beta & 0 & 0 \\
    \gamma \beta & \gamma & 0 & 0 \\
    0 & 0 & 1 & 0 \\
    0 & 0 & 0 & 1
    \end{bmatrix}.
    \]
    }
    \item The Lorentz transformations preserve the invariant interval:
    \eqbox{
    \[
    \Delta s^2 = x \cdot x = x'\cdot x'.
    \]
    }
\end{itemize}
}

\conceptbox{
\textbf{Applications:}
\begin{itemize}
    \item Lorentz transformations explain phenomena like time dilation, length contraction, and the relativity of simultaneity.
    \item They form the basis for defining and transforming four-vectors, such as four-momentum and four-velocity.
\end{itemize}
}


\subsection{The Relativistic Invariant Interval}
\conceptbox{
\textbf{Invariant Interval:}
The relativistic invariant interval is a spacetime quantity that remains unchanged under Lorentz transformations.
}
\eqbox{
\textbf{Invariant Interval:}
\[
\Delta s^2 = (c\Delta t)^2 - (\Delta x)^2 - (\Delta y)^2 - (\Delta z)^2
\]}
\conceptbox{
\textbf{Classifications of the Invariant Interval:}
\begin{itemize}
    \item \textbf{Time-like:} \( \Delta s^2 > 0 \), related to proper time:
    \eqbox{
    \[
    \Delta s^2 = c^2 \Delta \tau^2
    \]
    }
    \item \textbf{Space-like:} \( \Delta s^2 < 0 \), related to proper length:
    \eqbox{
    \[
    -\Delta s^2 = L^2
    \]
    }
    \item \textbf{Light-like:} \( \Delta s^2 = 0 \), events are connected by light:
    \eqbox{
    \[
    c^2\Delta t^2 = (\Delta x)^2 + (\Delta y)^2 + (\Delta z)^2
    \]
    }
\end{itemize}
}

\subsection{Relativistic Velocity Addition}
\conceptbox{
\textbf{Relativistic Velocity Addition:}
When combining velocities in different inertial frames, the relativistic formula must be used. Typically, these problems involve 3 objects and the following reference frames and velocities are defined:
\begin{itemize}
    \item \( S \): The stationary reference frame.
    \item \( S' \): The moving reference frame.
    \item \( u \): The velocity of an object relative to \( S \).
    \item \( v \): The velocity of \( S' \) relative to \( S \).
    \item \( u' \): The velocity of the object relative to \( S' \).
\end{itemize}
\eqbox{
\textbf{Relativistic Velocity Addition Formula:}
\[
u' = \frac{u + v}{1 + \frac{uv}{c^2}}
\]
Here, \( u' \) is the velocity of the object as measured in \( S' \), given the velocity \( u \) of the object relative to \( S \) and the velocity \( v \) of \( S' \) relative to \( S \).
}
}


\subsection{Four-Vectors in Special Relativity}
\conceptbox{
\textbf{Four-Vectors:}
Physical quantities in relativity are represented as four-vectors that transform under Lorentz transformations.
}
\eqbox{
\textbf{General Form of a Four-Vector:}
\[
a^\mu = (a^0, \vec{a}) = (a^0, a^1, a^2, a^3), \quad a^0 = c\Delta t
\]
}
\conceptbox{
\textbf{Dot Product of Four-Vectors:}
The invariant dot product between two four-vectors \( a^\mu \) and \( b^\mu \) is:
\eqbox{
\[
a \cdot b = a^0 b^0 - \vec{a} \cdot \vec{b}.
\]
}
This quantity is Lorentz invariant (unchanged under Lorentz transformations).

\textbf{Four-Position Dot Product:}
The four-position vector \( x^\mu \) is defined as \( x^\mu = (ct, \vec{x}) \), where \( t \) is time and \( \vec{x} \) is the spatial position. The dot product of \( x^\mu \) with itself gives the invariant interval:
\eqbox{
\[
x \cdot x = (ct)^2 - |\vec{x}|^2.
\]
}

\textbf{Four-Velocity Dot Product:}
The four-velocity vector \( u^\mu \) is defined as:
\eqbox{
\[
u^\mu = \frac{dx^\mu}{d\tau} = \gamma (c, \vec{v}),
\]
}
where \( \gamma = \frac{1}{\sqrt{1 - v^2/c^2}} \) is the Lorentz factor. The dot product of the four-velocity with itself is invariant and equals \( c^2 \):
\eqbox{
\[
u \cdot u = u^\mu u_\mu = c^2.
\]
}

\textbf{Four-Momentum Dot Product:}
The four-momentum vector \( p^\mu \) is defined as:
\eqbox{
\[
p^\mu = m u^\mu = \gamma m (c, \vec{v}),
\]
}
where \( m \) is the particle's rest mass. The dot product of the four-momentum vector with itself gives:
\eqbox{
\[
p \cdot p = p^\mu p_\mu = m^2 c^2.
\]
}
For massless particles, such as photons, \( p \cdot p = 0 \).

\textbf{Key Insights:}
\begin{itemize}
    \item The dot product between four-vectors remains invariant under Lorentz transformations, reflecting the relativistic consistency of spacetime intervals and physical quantities.
    \item The four-velocity and four-momentum vectors provide a direct link between classical mechanics and relativistic dynamics.
\end{itemize}
}


\subsection{Examples of Four-Vectors}
\conceptbox{
\textbf{Common Four-Vectors:}
\begin{itemize}
    \item \textbf{Four-Position:} 
    \eqbox{
    \[
    x^\mu = (ct, x, y, z)
    \]
    }
    \item \textbf{Four-Velocity:} 
    \eqbox{
    \[
    u^\mu = \frac{dx^\mu}{d\tau} = \gamma (c, \vec{v}), \quad \gamma = \frac{1}{\sqrt{1 - v^2/c^2}}
    \]
    }
    \item \textbf{Four-Momentum:} 
    \eqbox{
    \[
    p^\mu = m u^\mu = \gamma m (c, \vec{v})
    \]
    }
    \eqbox{
    For massless particles, like photons:
    \[
    p \cdot p = 0, \quad E = |\vec{p}|c
    \]
    }
\end{itemize}
}

\subsection{Relativistic Dynamics and Conservation Laws}

\conceptbox{
\textbf{Relativistic Conservation Laws:}
The principles of conservation of energy and momentum extend to special relativity, incorporating the total energy (rest energy + kinetic energy) and relativistic momentum.
}

\conceptbox{
    \textbf{Relativistic Total Energy:}
    The total energy \( E \) includes both the rest energy and the relativistic kinetic energy:
    \eqbox{
    \[
    E_{tot} = \gamma m_0 c^2.
    \]
    }
    \textbf{Relativistic Kinetic Energy:} 
    The kinetic energy is given by
    \eqbox{
        \[
            KE = (\gamma - 1)m_0c^2
        \]
    }
    \textbf{Rest Energy:}
    The rest energy is given by
    \eqbox{\[ E_0 = m_0 c^2 \]} where \( m_0 \) is the rest mass.
    \vspace{1em}

    In any physical process, the total energy is conserved.
}
\conceptbox{
    \textbf{Relativistic Momentum:}
    \begin{itemize}
        \item The relativistic momentum is defined as:
        \eqbox{
        \[
        \vec{p} = \gamma m_0 \vec{v},
        \]
        }
        where \( \gamma = \frac{1}{\sqrt{1 - v^2/c^2}} \).
        \item Both the magnitude and direction of total momentum in an isolated system are conserved.
    \end{itemize}
}

\conceptbox{
    \textbf{Four-Momentum Conservation:}
    \begin{itemize}
        \item Energy and momentum form components of the four-momentum vector:
        \eqbox{
        \[
        p^\mu = \left( \frac{E}{c}, \vec{p} \right).
        \]
        }
        \item In all inertial frames, the conservation of four-momentum holds:
        \eqbox{
        \[
        \sum p^\mu_{\text{initial}} = \sum p^\mu_{\text{final}}.
        \]
    }
    \end{itemize}
}

\conceptbox{
    \textbf{Relativistic Collisions:}
    \begin{itemize}
        \item Total energy and total momentum are conserved in collisions, including elastic and inelastic cases.
        \item Unlike classical mechanics, the concept of conserved "mass" is not generally applicable. For example:
        \begin{itemize}
            \item In pair annihilation, rest mass converts to energy.
            \item In particle creation, energy converts into rest mass.
        \end{itemize}
    \end{itemize}
}

\eqbox{
\textbf{Energy-Momentum Relation:}
\[
E^2 = (pc)^2 + (m_0c^2)^2
\]
This relation connects the total energy \( E \), the momentum \( p \), and the rest mass \( m_0 \) of a particle.
}

\eqbox{
\textbf{Invariant Four-Momentum Magnitude:}
\[
p^\mu \cdot p^\mu = m_0^2 c^2
\]
This invariant magnitude holds in all inertial frames, ensuring the consistency of relativistic conservation laws.
}
\newpage
\section{Quantum Mechanics}

\subsection{Quantum Theory of Light}

\subsubsection{Blackbody Radiation}

\conceptbox{
\textbf{Blackbody Radiation:}
A blackbody is an idealized object that absorbs all incident electromagnetic radiation and re-emits it as a function of its temperature. The spectral intensity of radiation emitted by a blackbody is described by Planck's Law.
}

\conceptbox{
\textbf{Key Challenge:}
Classical physics predicted the "ultraviolet catastrophe," where energy radiated at high frequencies diverged. Planck resolved this by introducing the concept of quantized energy:
\eqbox{
\[
E = nhf, \quad n = 1, 2, 3, \dots
\]
}
This assumption restricted the energy of oscillators to discrete levels, successfully explaining blackbody radiation.
}

\conceptbox{
\textbf{Planck's Law:}
Describes the spectral radiance of a blackbody at a given temperature \( T \):
\eqbox{
\[
I(\lambda, T) = \frac{2\pi hc^2}{\lambda^5}\frac{1}{e^{hc/\lambda k_BT} - 1}, \quad I(f, T) = \frac{8 \pi h f^3}{c^3}\frac{1}{e^{hf/k_BT} - 1}.
\]
}
Here:
\begin{itemize}
    \item \( I(\lambda, T) \): Intensity as a function of wavelength \( \lambda \).
    \item \( I(f, T) \): Intensity as a function of frequency \( f \).
    \item \( h \): Planck's constant.
    \item \( c \): Speed of light.
    \item \( k_B \): Boltzmann constant.
\end{itemize}
}

\conceptbox{
\textbf{Stefan-Boltzmann Law:}
The total power radiated per unit area of a blackbody is proportional to the fourth power of its absolute temperature:
\eqbox{
\[
P = \sigma T^4, \quad \sigma = \frac{2\pi^5 k_B^4}{15c^2h^3} \approx 5.67 \times 10^{-8} \, \text{W/m}^2\cdot \text{K}^4.
\]
}
Here:
\begin{itemize}
    \item \( P \): Power per unit area emitted by the blackbody.
    \item \( \sigma \): Stefan-Boltzmann constant.
\end{itemize}
}

\conceptbox{
\textbf{Wien's Displacement Law:}
The wavelength \( \lambda_{\text{max}} \) at which the blackbody radiation is most intense is inversely proportional to the temperature:
\eqbox{
\[
\lambda_{\text{max}} = \frac{b}{T}, \quad b \approx 2.897 \times 10^{-3} \, \text{m} \cdot \text{K}.
\]
}
Here:
\begin{itemize}
    \item \( \lambda_{\text{max}} \): Wavelength at peak intensity.
    \item \( T \): Absolute temperature.
    \item \( b \): Wien's constant.
\end{itemize}
}

\conceptbox{
\textbf{Key Insights:}
\begin{itemize}
    \item Planck's quantization hypothesis was a pivotal step toward quantum mechanics.
    \item The Stefan-Boltzmann law integrates Planck's law over all wavelengths to find total radiated power.
    \item Wien's displacement law helps identify the peak emission wavelength of a blackbody at a given temperature.
\end{itemize}
}

\subsubsection{Photoelectric Effect}

\conceptbox{
\textbf{Photoelectric Effect:}
The photoelectric effect describes the emission of electrons from a material when light of sufficient frequency shines on it. The energy of the emitted electrons depends on the frequency of the light, not its intensity, providing evidence for the particle-like nature of light.
}

\conceptbox{
\textbf{Photoelectric Effect: Classical Predictions vs. Experimental Results}
\vspace{0.2cm}
\small
\begin{center}
\renewcommand{\arraystretch}{1.2} % Adjust row spacing
\begin{tabular}{|p{2.8cm}|p{5cm}|p{5cm}|}
\hline
\textbf{Aspect} & \textbf{Classical Prediction} & \textbf{Experimental Result} \\\hline
\textbf{Dependence on Intensity} & Increasing light intensity increases electron energy. & Electron energy is independent of intensity; only the number of electrons increases. \\\hline
\textbf{Dependence on Frequency} & Electron emission occurs at any frequency with enough intensity. & Electrons are emitted only if frequency exceeds a threshold \( f_c \). \\\hline
\textbf{Energy of Emitted Electrons} & Energy of electrons depends on light intensity. & Electron energy depends linearly on light frequency: \( KE_{\text{max}} = hf - \phi \). \\\hline
\textbf{Time Delay} & Emission occurs after measurable delay for low-intensity light. & Electrons are emitted instantaneously, regardless of light intensity. \\\hline
\end{tabular}
\end{center}
\normalsize
}

\conceptbox{
\textbf{Einstein's Photoelectric Equation:}
The energy of the incident photons is used to overcome the work function (\( \phi \)) of the material and provide the kinetic energy (\( KE_{max} \)) to the ejected electrons:
\eqbox{
\[
hf = \phi + KE_{max},
\]
}
where:
\begin{itemize}
    \item \( h \): Planck's constant.
    \item \( f \): Frequency of the incident light.
    \item \( \phi \): Work function (minimum energy needed to eject an electron).
    \item \( KE_{max} \): Maximum kinetic energy of the emitted electron.
\end{itemize}
}

\conceptbox{
\textbf{Threshold Frequency:}
Electrons are only emitted if the frequency of light \( f \) exceeds the threshold frequency \( f_c \):
\eqbox{
\[
hf_c = \phi.
\]
}
If \( f < f_c \), no electrons are emitted regardless of the light's intensity.
}

\conceptbox{
\textbf{Stopping Potential:}
To measure \( KE_{max} \), a stopping potential \( V_s \) is applied to halt the emitted electrons:
\eqbox{
\[
KE_{max} = eV_s,
\]
}
where \( e \) is the elementary charge.
}

\conceptbox{
\textbf{Key Observations:}
\begin{itemize}
    \item The kinetic energy of emitted electrons depends on the light's frequency, not its intensity.
    \item There is a threshold frequency below which no electrons are emitted.
    \item Increasing light intensity increases the number of emitted electrons but not their kinetic energy.
    \item The stopping potential is independent of light intensity and depends solely on \( KE_{max} \).
\end{itemize}
}

\conceptbox{
\textbf{Experimental Verification:}
The photoelectric effect was experimentally validated by observing:
\begin{itemize}
    \item The emission of electrons from metals when illuminated by ultraviolet light.
    \item A linear relationship between the frequency of light and \( KE_{max} \).
\end{itemize}
This experiment strongly supported Einstein's photon theory of light, which extended Planck's quantization hypothesis to all electromagnetic radiation.
}

\conceptbox{
\textbf{Applications:}
\begin{itemize}
    \item Photoelectric cells (e.g., solar panels) convert light into electricity.
    \item Light sensors and detectors in various scientific instruments.
    \item Validation of quantum theory and photon-based models of light.
\end{itemize}
}
\newpage
\subsubsection{Compton Scattering}

\conceptbox{
\textbf{Compton Scattering:}
Compton scattering describes the interaction between high-energy photons (e.g., X-rays) and electrons, leading to a measurable change in the photon's wavelength. This phenomenon provides strong evidence for the particle-like properties of light.
}

\conceptbox{
\textbf{Compton Wavelength Shift:}
The change in the wavelength of the photon after scattering is given by:
\eqbox{
\[
\lambda' - \lambda = \frac{h}{m_ec}(1 - \cos\theta),
\]
}
where:
\begin{itemize}
    \item \( \lambda \): Initial wavelength of the photon.
    \item \( \lambda' \): Wavelength of the scattered photon.
    \item \( h \): Planck's constant.
    \item \( m_e \): Mass of the electron.
    \item \( c \): Speed of light.
    \item \( \theta \): Angle of scattering (angle between the initial and scattered photon).
\end{itemize}
}

\conceptbox{
\textbf{Energy Relation:}
The relationship between the photon's initial and final energies is:
\eqbox{
\[
\frac{1}{E'} - \frac{1}{E} = \frac{1 - \cos\theta}{m_ec^2},
\]
}
where:
\begin{itemize}
    \item \( E \): Initial photon energy (\( E = hf \)).
    \item \( E' \): Energy of the scattered photon.
    \item \( m_ec^2 \): Rest energy of the electron.
\end{itemize}
}

\conceptbox{
\textbf{Key Observations:}
\begin{itemize}
    \item The scattered photon has lower energy and longer wavelength than the incident photon, consistent with energy conservation.
    \item The shift in wavelength depends only on the scattering angle \( \theta \) and fundamental constants, not the material or initial photon energy.
    \item Demonstrates that photons carry momentum, supporting their particle-like nature.
\end{itemize}
}

\conceptbox{
\textbf{Experimental Verification:}
\begin{itemize}
    \item Arthur Compton's experiments in 1923 confirmed the theoretical prediction of the wavelength shift.
    \item Results agreed precisely with the derived equation for the Compton wavelength shift, providing strong evidence for quantum theory.
\end{itemize}
}

\conceptbox{
\textbf{Applications:}
\begin{itemize}
    \item Used in X-ray and gamma-ray spectroscopy to study material properties.
    \item Plays a key role in medical imaging techniques such as Compton scattering tomography.
    \item Foundational to understanding photon interactions in particle physics and astrophysics.
\end{itemize}
}

\subsubsection{Absorption and Emission Spectrum}
\conceptbox{
\textbf{Absorption and Emission Spectrum:}
Atoms interact with light through the absorption or emission of photons. This process leads to the formation of distinct spectra that depend on the energy differences between the quantized energy levels of the atom.

\textbf{Key Concepts:}
\begin{itemize}
    \item \textbf{Absorption Spectrum:}
    When light passes through a gas, atoms absorb photons of specific energies corresponding to transitions between lower and higher energy levels. The resulting spectrum shows dark lines at these wavelengths against a continuous background.
    \item \textbf{Emission Spectrum:}
    Excited atoms release energy by emitting photons as electrons transition from higher to lower energy levels. The emitted light forms bright lines at specific wavelengths.
\end{itemize}
}

\conceptbox{
\textbf{Energy of a Photon:}
The energy difference between two levels determines the energy (and wavelength) of absorbed or emitted photons:
\eqbox{
\[
E_{\text{photon}} = hf = \frac{hc}{\lambda} = E_i - E_f,
\]
}
where:
\begin{itemize}
    \item \( E_{\text{photon}} \): Energy of the photon.
    \item \( E_i \): Energy of the initial (higher) state.
    \item \( E_f \): Energy of the final (lower) state.
    \item \( \lambda \): Wavelength of the absorbed/emitted light.
    \item \( h \): Planck's constant, \( c \): Speed of light.
\end{itemize}
}

\conceptbox{
\textbf{Hydrogen Atom Spectrum:}
For hydrogen, the energy levels are given by:
\eqbox{
\[
E_n = -\frac{13.6 \, \text{eV}}{n^2}, \quad n = 1, 2, 3, \dots
\]
}
\textbf{Spectral Series:}
The transitions between energy levels produce specific series in the hydrogen spectrum:
\begin{itemize}
    \item \textbf{Lyman Series:} \( n_f = 1 \), transitions in the ultraviolet range.
    \item \textbf{Balmer Series:} \( n_f = 2 \), transitions in the visible range.
    \item \textbf{Paschen Series:} \( n_f = 3 \), transitions in the infrared range.
\end{itemize}
}

\conceptbox{
\textbf{Key Observations:}
\begin{itemize}
    \item Absorption spectra are unique to each element and are used in spectroscopy to identify substances.
    \item Emission spectra arise from electron transitions and are used to study excited atoms and astrophysical objects.
    \item The energy and wavelength of absorbed/emitted photons provide direct evidence for quantized energy levels in atoms.
\end{itemize}
}

\newpage
\subsubsection{Wave-Particle Duality}

\conceptbox{
\textbf{Wave-Particle Duality:}
Light and matter exhibit dual behavior, displaying characteristics of both waves and particles depending on the experimental conditions.
}
\conceptbox{
\textbf{Key Features:}
\begin{itemize}
    \item Travels as a wave, interacting with itself, and demonstrates phenomena such as interference and diffraction.
    \item Interacts as a particle, transferring discrete packets of energy, known as photons or quanta.
    \item Confirmed by experiments:
    \begin{itemize}
        \item \textbf{Double-Slit Experiment:} Demonstrates interference patterns for light and electrons, showing wave-like behavior.
        \item \textbf{Photoelectric Effect:} Demonstrates the particle nature of light, as photons eject electrons from a material.
        \item \textbf{Blackbody Radiation:} Explained by Planck's quantization hypothesis, resolving the ultraviolet catastrophe.
        \item \textbf{Compton Scattering:} Demonstrates photon momentum through scattering with electrons.
    \end{itemize}
\end{itemize}
\eqbox{
\textbf{De Broglie Wavelength:} \( \lambda = \displaystyle\frac{h}{p} = \displaystyle\frac{h}{mv} \)
}

\eqbox{
\textbf{Wave Number:} \( k = \displaystyle\frac{2\pi}{\lambda} \)
}
}

\subsubsection{Heisenberg Uncertainty Principle}

\conceptbox{
\textbf{Heisenberg Uncertainty Principle:}
A fundamental principle of quantum mechanics stating that certain pairs of physical observables cannot be simultaneously measured with arbitrary precision.
}

\conceptbox{
\textbf{Key Relations:}
\begin{itemize}
    \item \textbf{Position-Momentum Uncertainty:}
    \eqbox{
    \[
    \Delta x \Delta p \geq \frac{\hbar}{2},
    \]
    }
    where \( \Delta x \) and \( \Delta p \) are the standard deviations of position and momentum, respectively.
    \item \textbf{Energy-Time Uncertainty:}
    \eqbox{
    \[
    \Delta E \Delta t \geq \frac{\hbar}{2},
    \]
    }
    where \( \Delta E \) and \( \Delta t \) are the uncertainties in energy and time, respectively.
\end{itemize}
}

\conceptbox{
\textbf{Variance-Squared Average Relationship:}
The uncertainty in a measurable quantity \( A \) is defined as:
\eqbox{
\[
\Delta A = \sqrt{\langle A^2 \rangle - \langle A \rangle^2},
\]
}
where \( \langle A \rangle \) is the expectation value, and \( \langle A^2 \rangle \) is the expectation value of \( A^2 \).
}

\conceptbox{
\textbf{Implications:}
\begin{itemize}
    \item Establishes the probabilistic nature of quantum mechanics, replacing deterministic classical physics.
    \item Demonstrates the impossibility of assigning definite trajectories to particles as in classical mechanics.
\end{itemize}
}

\subsubsection{Group and Phase Velocities}

\conceptbox{
\textbf{Group and Phase Velocities:}
These describe two distinct velocities associated with wave propagation. While group velocity represents the speed of the overall wave packet (and the particle), phase velocity corresponds to the speed of individual wave crests.
}

\conceptbox{
\textbf{Group Velocity (\( v_g \)):}
The velocity of the wave packet, which often represents the velocity of energy or information transfer. It is defined as:
\eqbox{
\[
v_g = \frac{d\omega}{dk},
\]
}
where:
\begin{itemize}
    \item \( \omega \): Angular frequency of the wave.
    \item \( k \): Wave number (\( k = 2\pi/\lambda \)).
\end{itemize}
\textbf{Key Features:}
\begin{itemize}
    \item For non-dispersive media (constant \( v \)), \( v_g = v_p \).
    \item In quantum mechanics, \( v_g \) is often associated with the particle's velocity.
    \item Relevant for wave packets, where multiple frequencies combine to form a localized pulse.
\end{itemize}
}

\conceptbox{
\textbf{Phase Velocity (\( v_p \)):}
The velocity of individual wave crests, which is distinct from the velocity of the overall wave packet. It is defined as:
\eqbox{
\[
v_p = \frac{\omega}{k},
\]
}
where:
\begin{itemize}
    \item \( \omega \): Angular frequency.
    \item \( k \): Wave number.
\end{itemize}
\textbf{Key Features:}
\begin{itemize}
    \item \( v_p \) can exceed the speed of light (\( c \)) in certain media without violating causality.
    \item Represents the velocity of harmonic waves in a wave train.
    \item In dispersive media, \( v_p \neq v_g \).
\end{itemize}
}

\conceptbox{
\textbf{Relation Between Group and Phase Velocities:}
In dispersive media, the group velocity depends on the phase velocity:
\eqbox{
\[
v_g = v_p - \lambda \frac{dv_p}{d\lambda},
\]
}
where:
\begin{itemize}
    \item \( \lambda \): Wavelength of the wave.
    \item \( \frac{dv_p}{d\lambda} \): Rate of change of phase velocity with respect to wavelength.
\end{itemize}
}

\conceptbox{
\textbf{Key Observations:}
\begin{itemize}
    \item \( v_g \) is the velocity of energy or particle propagation, making it more relevant for physical phenomena.
    \item \( v_p \) can provide insights into the nature of wave propagation in a medium.
    \item The relationship between \( v_g \) and \( v_p \) becomes significant in understanding wave dispersion.
\end{itemize}
}

\conceptbox{
\textbf{Applications:}
\begin{itemize}
    \item Group velocity is crucial in quantum mechanics for describing particle motion through the de Broglie wavelength.
    \item Phase velocity plays a role in characterizing electromagnetic waves in media, including fiber optics and plasma physics.
    \item Both are essential for understanding dispersion in waves, which occurs in optics, acoustics, and quantum systems.
\end{itemize}
}



\subsection{Schrödinger Equation}

\subsubsection{Overview}
\conceptbox{
\textbf{Schrödinger Equation:} Governs the quantum mechanical behavior of particles. It describes how the wave function of a particle evolves under the influence of potential energy.

\textbf{General Forms:}
\begin{itemize}
    \item \textbf{1D Time-Independent Schrödinger Equation:}
    \eqbox{
    \[
    -\frac{\hbar^2}{2m} \frac{d^2\psi(x)}{dx^2} + V(x)\psi(x) = E\psi(x).
    \]}
    \item \textbf{3D Schrödinger Equation (spherical coordinates):}
    \eqbox{
    \[
    -\frac{\hbar^2}{2m} \nabla^2 \psi(r, \theta, \phi) + V(r)\psi(r, \theta, \phi) = E\psi(r, \theta, \phi).
    \]}
    The solution is separable:
    \eqbox{
    \[
    \psi(r, \theta, \phi) = R(r)Y(\theta, \phi),
    \]}
    where \( Y(\theta, \phi) \) are spherical harmonics.
\end{itemize}
}

\subsubsection{Normalization of the Wave Function}
\conceptbox{
\textbf{Normalization Condition:}
The probability of finding a particle in the entire space must be equal to 1. The wave function \( \psi(x) \) (or \( \psi(r, \theta, \phi) \) in 3D) is normalized if:
\eqbox{
\[
\int_{-\infty}^\infty |\psi(x)|^2 dx = 1 \quad \text{(1D)} \quad \text{or} \quad \int |\psi(r, \theta, \phi)|^2 d\tau = 1 \quad \text{(3D)}.
\]
}
Here:
\begin{itemize}
    \item \( |\psi(x)|^2 \) is the probability density in 1D.
    \item \( |\psi(r, \theta, \phi)|^2 \) is the probability density in 3D.
    \item \( d\tau = r^2 \sin\theta \, dr \, d\theta \, d\phi \) is the volume element in spherical coordinates.
\end{itemize}

\textbf{Key Insight:}
If the wave function is not normalized, it can be normalized by multiplying it by a constant \( A \), such that:
\eqbox{
\[
\psi_{\text{norm}}(x) = A\psi(x), \quad A = \frac{1}{\sqrt{\int |\psi(x)|^2 dx}}.
\]}
}


\subsubsection{Different Potentials}
\conceptbox{
\textbf{Free Particle:}
\begin{itemize}
    \item For \( V(x) = 0 \), solutions are plane waves:
    \eqbox{
    \[
    \psi(x) = Ae^{ikx} + Be^{-ikx},
    \]}
    where \( k = \frac{p}{\hbar} \), \( \omega = \frac{E}{\hbar} \), and \(p = \hbar k = \sqrt{2mE}\).
    \item \(e^{ikx}\) is right moving plane wave and \(e^{-ikx}\) is left moving plane wave.
    \item For full time dependent, \(\Psi(x,t) = \psi(x)e^{-i\omega t}\)
    \item Wave function is not normalizable!
\end{itemize}
}

\conceptbox{
\textbf{Particle in a Box/Infinite Square Well Potential:}
\begin{itemize}
    \item For \( V(x) = 0 \) inside the box and \( V(x) = \infty \) outside:
    \begin{itemize}
        \item General Solution 1D:
            \eqbox{
                \[
                    \psi_n(x) = A\sin\left(\frac{n\pi x}{L}\right)
                \]
            }
        \item General Solution 3D:
            \eqbox{
                \[
                \psi_{n_1, n_2, n_3} = A\sin(\frac{n_1 \pi x}{L})\sin(\frac{n_2 \pi y}{L})\sin(\frac{n_3 \pi z}{L})
                \]
            }
        \item Particular Solution 1D:
            \eqbox{
            \[
            \psi_n(x) = \sqrt{\frac{2}{L}} \sin\left(\frac{n\pi x}{L}\right), \quad E_n = \frac{n^2\pi^2\hbar^2}{2mL^2} = \frac{n^2 h^2}{8mL^2}
            \]}
        \item Particular Solution 3D:
            \eqbox{
                \[
                \psi_{n_1, n_2, n_3} = \left(\frac{2}{L}\right)^{3/2}\sin(\frac{n_1 \pi x}{L})\sin(\frac{n_2 \pi y}{L})\sin(\frac{n_3 \pi z}{L})
                \]
                \[
                \quad E_n = \frac{\hbar ^2 \pi^2}{2mL^2}(n_1^2 + n_2^2 + n_3^2)
                \]
            }
    \end{itemize}
\end{itemize}
}


\conceptbox{
\textbf{Step Potential:}
\eqbox{
\[
V(x) = 
\begin{cases} 
0, & x < 0 \\ 
U_0, & x \geq 0 
\end{cases}
\]
}
The particle's behavior depends on whether its energy \( E \) is greater than or less than \( U_0 \).
}

\conceptbox{
\subsubsection*{Case 1: \( E > U_0 \)}

\textbf{Wave Function Solutions:}
\eqbox{
\begin{itemize}
    \item \( x < 0 \): \( \psi(x) = Ae^{ik_0x} + Be^{-ik_0x}, \quad k_0 = \sqrt{\frac{2mE}{\hbar^2}} \)
    \item \( x \geq 0 \): \( \psi(x) = Ce^{ik_1x}, \quad k_1 = \sqrt{\frac{2m(E - U_0)}{\hbar^2}} \)
\end{itemize}
}

\textbf{Reflection and Transmission Coefficients:}
\eqbox{
\[
R = \left( \frac{k_0 - k_1}{k_0 + k_1} \right)^2, \quad T = \frac{4k_0k_1}{(k_0 + k_1)^2}, \quad R + T = 1
\]
}
}


\conceptbox{
\subsubsection*{Case 2: \( E < U_0 \)}
\textbf{Wave Function Solutions:}
\eqbox{
\begin{itemize}
    \item \( x < 0 \): \( \psi(x) = Ae^{ik_0x} + Be^{-ik_0x}, \quad k_0 = \sqrt{\frac{2mE}{\hbar^2}} \)
    \item \( x \geq 0 \): \( \psi(x) = De^{-\kappa x}, \quad \kappa = \sqrt{\frac{2m(U_0 - E)}{\hbar^2}} \)
\end{itemize}
}

\textbf{Tunneling Insight:}
The particle cannot propagate in the \( x \geq 0 \) region but has a finite probability of being found in the barrier due to tunneling.
}

\conceptbox{
\textbf{Key Insights:}
\begin{itemize}
    \item For \( E > U_0 \), the particle has probabilities of transmission and reflection at the step.
    \item For \( E < U_0 \), the particle exhibits tunneling with an exponentially decaying wave function in the barrier.
\end{itemize}
}


\subsection{Hydrogen Atom}

\subsubsection{Wave Functions}
\conceptbox{
\textbf{Wave Functions:}
\eqbox{
\[
\psi(r, \theta, \phi) = R_{nl}(r)Y_l^m(\theta, \phi),
\]}
where \( R_{nl}(r) \) is the radial wave function, and \( Y_l^m(\theta, \phi) \) are spherical harmonics.

\textbf{Radial Probability Distribution:}
The probability density \( P(r) \) is given by:
\eqbox{
\[
P(r) = r^2|R_{nl}(r)|^2.
\]}
}

\subsubsection{Quantum Numbers}
\conceptbox{
\textbf{Quantum Numbers:}
\begin{itemize}
    \item \textbf{Principal quantum number (\( n \)):} Determines energy level (\( n = 1, 2, 3, \dots \)).
    \begin{itemize}
        \item Tells us the \textbf{energy} of the Hydrogen atom
        \eqbox{
            \[
                E(\psi_{n, \ell, m}) = -\frac{13.6\text{ eV}}{n^2}
            \]
        }
    \end{itemize}
    \item \textbf{Orbital angular momentum quantum number (\( \ell \)):} Determines the shape of the orbital (\( \ell = 0, 1, \dots, n-1 \)).
    \begin{itemize}
        \item Tells us the \textbf{total angular momentum} of the Hydrogen atom
        \eqbox{
            \[
                L(\psi_{n, \ell, m}) = \hbar \sqrt{\ell(\ell + 1)}
            \]
        }
    \end{itemize}
    \item \textbf{Magnetic quantum number (\( m \)):} Determines the orientation (\( m = -\ell, -\ell+1, \dots, \ell \)).
    \begin{itemize}
        \item Tells us the \textbf{z-component of angular momentum} of the Hydrogen atom
        \eqbox{
            \[
                L_z(\psi_{n, \ell, m}) = \hbar m
            \]
        }
    \end{itemize}
    \item \textbf{Spin quantum number (\( m_s \)):} Intrinsic angular momentum (\( m_s = \pm \frac{1}{2} \)).
    \begin{itemize}
        \item Has to be rotated \textbf{720 degrees} for it to return to original state
        \item Fermions have half integer spin (electrons, muons, protons, neutrons, etc.)
        \item Bosons have integer spin (photons, W and Z bosons, Higgs boson, etc.)
        \item Stern-Gerlach Experiment
    \end{itemize}
\end{itemize}
}

\subsubsection{Multi-Electron Atoms and the Pauli Exclusion Principle}

\conceptbox{
\textbf{Multi-Electron Atoms:}
In atoms with more than one electron, the arrangement of electrons is determined by their energies and quantum states:
\begin{itemize}
    \item Electrons occupy orbitals starting from the lowest energy levels, but no more than two electrons can share the same orbital.
    \item Inner electrons shield outer electrons from the full nuclear charge, affecting orbital energies.
    \item Example: In Potassium (\( Z = 19 \)), the electron configuration is \( 1s^2 2s^2 2p^6 3s^2 3p^6 4s^1 \). The last electron occupies the \( 4s \) orbital instead of \( 3d \) due to shielding.
\end{itemize}
}

\conceptbox{
\textbf{Pauli Exclusion Principle:}
No two electrons in an atom can occupy the same quantum state. This means:
\begin{itemize}
    \item No two electrons can have identical quantum numbers (\( n, l, m, m_s \)).
    \item This principle determines the filling of electron orbitals and gives rise to electron configurations.
\end{itemize}
}

\conceptbox{
\textbf{Electron Configuration:}
The distribution of electrons in orbitals follows these rules:
\begin{itemize}
    \item Electrons are added one at a time to the lowest energy subshell available.
    \item The periodic table reflects the outer electron configurations, which dictate chemical properties.
\end{itemize}
}

\section{Statistical Mechanics}

\subsection{Microscopic vs. Macroscopic States}
\conceptbox{
\textbf{Microscopic States:} 
The state of a system described by the position and momentum of every particle. For \( N \) particles, this requires \( 6N \) values (3 spatial coordinates and 3 momentum components per particle).

\textbf{Macroscopic States:} 
When \( N \) is large (e.g., Avogadro's number), it is impractical to track each particle. Instead, systems are described using averaged quantities like energy, volume, and temperature.

\textbf{Example:} 
For 10 coins:
\begin{itemize}
    \item Total \textbf{microstates:} \( 2^{10} = 1024 \).
    \item Total \textbf{macrostates:} 11 (e.g., all heads, 1 tail and 9 heads, etc.).
\end{itemize}
}

\subsection{Multiplicity and Degeneracy}
\conceptbox{
\textbf{Multiplicity (\( g \)):} 
The number of microstates corresponding to a macrostate. Calculated as:
\eqbox{
\[
g = \binom{N}{k} = \frac{N!}{k!(N-k)!},
\]
}
where \( k \) is the number of particles in a specific state.

\textbf{Example:}
For 10 coins with 3 tails and 7 heads:
\[
g = \binom{10}{3} = \frac{10!}{3!7!} = 120.
\]

\textbf{Degeneracy:} 
The number of quantum states associated with a given energy level.

\textbf{Microcanonical Ensemble:} 
A set of all microstates with the same energy, where each is equally probable.
}

\subsection{Statistical Distributions}
\conceptbox{
\textbf{Probability Distributions:}
The likelihood of finding a particle in a state with energy \( E_i \) depends on temperature (\( T \)) and energy. The three primary distributions are:

\begin{itemize}
    \item \textbf{Maxwell-Boltzmann (Distinguishable Particles):}
    Applicable when \( T \gg T_c \) (thermal energy \( kT \) is much larger than quantum energy level spacing) and particle density is low.
    \eqbox{
    \[
    f(E) = f_B(E) = Ae^{-E/kT}
    \]
    }
    \item \textbf{Bose-Einstein (Indistinguishable Bosons):}
    Below the critical temperature (\( T_c \)), a Bose-Einstein condensate forms as many bosons occupy the ground state.
    \eqbox{
    \[
    f(E) = f_{BE}(E) = \frac{1}{e^\alpha e^{E / k_BT} - 1}.
    \]
    }
    \item \textbf{Fermi-Dirac (Indistinguishable Fermions):}
    Governed by the Pauli exclusion principle, where no two fermions occupy the same quantum state.
    \eqbox{
    \[
    f(E) = f_{FD}(E) = \frac{1}{e^\alpha e^{E / k_BT} + 1}.
    \]
    }
\end{itemize}
}
\conceptbox{
\textbf{Occupation Number:}
\begin{itemize}
    \item The \textbf{occupation number}, \( n(E)dE \), represents the average number of particles in a quantum state with energy between \( E \) and \( E + dE \).
    \item It is calculated as:
    \eqbox{
    \[
    n(E)dE = f(E)g(E)dE,
    \]
    }
    where:
    \begin{itemize}
        \item \( f(E) \) is the probability distribution (Maxwell-Boltzmann, Bose-Einstein, or Fermi-Dirac).
        \item \( g(E)dE \) is the degeneracy, or the number of states available in the energy range \( E \) to \( E + dE \).
    \end{itemize}
\end{itemize}
}

\subsection{Normalization Conditions}
\conceptbox{
\textbf{Normalization Conditions:}
\begin{itemize}
    \item The total number of particles in the system:
    \eqbox{
    \[
    N = \int f(E)g(E)dE,
    \]
    }
    where \( f(E) \) is the distribution function, and \( g(E) \) is the density of states.

    \item The total energy of the system:
    \eqbox{
    \[
    E_{\text{tot}} = \int f(E)g(E)E \, dE.
    \]
    }
\end{itemize}
}

\subsection{Maxwell-Boltzmann Distribution}
\conceptbox{
\textbf{Maxwell-Boltzmann Distribution:}
Describes the distribution of particles in a system of distinguishable particles.

\textbf{Degeneracy of States:}
For an ideal monatomic gas:
\eqbox{
\[
g(p)dp = 4\pi p^2 dp \to g(E)dE = 4\pi m \sqrt{2mE}dE
\]
}
where:
\begin{itemize}
    \item \( m \): Particle mass.
    \item \( E \): Energy of the particle.
\end{itemize}

\textbf{Key Properties:}
\begin{itemize}
    \item Valid for classical gases where quantum effects are negligible.
    \item Particles are distinguishable.
    \item No restriction on the number of particles in a given state.
\end{itemize}

\textbf{Applications:}
Used to describe the behavior of ideal gases in classical thermodynamics and kinetic theory.
}

\subsection{Bose-Einstein Distribution}
\conceptbox{
\textbf{Bose-Einstein Distribution:}
Describes indistinguishable bosons, such as photons, which can occupy the same quantum state.

\textbf{Degeneracy of States for Bosons:}
\eqbox{
\[
g(n)dn = \frac{1}{8}4\pi n^2 dn \to g(E)dE = \frac{\pi}{4 E_0^{3/2}}\sqrt E dE
\]
}
where \( E_0 = \displaystyle\frac{\hbar^2 \pi^2}{2m L^2} \)

\textbf{Key Properties:}
\begin{itemize}
    \item Bosons can occupy the same quantum state.
    \item Leads to phenomena like Bose-Einstein condensation at low temperatures.
\end{itemize}

\textbf{Applications:}
Used to describe systems like photons (blackbody radiation) and liquid helium (superfluidity).
}

\subsection{Bose-Einstein Condensation}
\conceptbox{
\textbf{Bose-Einstein Condensation:}
Occurs when a system of bosons is cooled below a critical temperature (\( T_c \)), causing many particles to occupy the ground state.

\textbf{Key Relations:}
\begin{itemize}
    \item Critical temperature:
    \eqbox{
    \[
    T_c = \left(\frac{h^2}{2\pi mk}\right) \left(\frac{N}{V\zeta(3/2)}\right)^{2/3}.
    \]
    }
    where \(\zeta(3/2) = 2.612\)
    \item Fraction in the ground state:
    \eqbox{
    \[
    \frac{N_0}{N} = 1 - \left(\frac{T}{T_c}\right)^{3/2}.
    \]
    }
\end{itemize}
}

\subsection{Fermi-Dirac Distribution}
\conceptbox{
\textbf{Fermi-Dirac Distribution:}
Describes indistinguishable fermions (particles with half-integer spin) that obey the Pauli Exclusion Principle.

\textbf{Degeneracy of States for Fermions:}
\eqbox{
\[
g(E)dE = \frac{\pi}{4 E_0^{3/2}}\sqrt E dE = 8\sqrt{2}\pi \left(\frac{m^{3/2}V}{h^3}\right)\sqrt E dE
\]
}
where:
\begin{itemize}
    \item \( V \): Volume of the system.
    \item \( m \): Mass of the particle.
    \item \(E_0 = \displaystyle\frac{h^2}{8mL^2}\)
\end{itemize}

\textbf{Key Properties:}
\begin{itemize}
    \item Fermions cannot occupy the same quantum state (Pauli Exclusion Principle).
    \item At absolute zero, all states up to the Fermi energy (\( E_F \)) are occupied, while higher energy states are empty.
    \begin{itemize}
        \item The Fermi energy forms the boundary between occupied and unoccupied states
        \item at \(T = 0\), all states with energies below \( E_F \) are occupied, and those above are not
        \eqbox{
            \[
                E_F = \displaystyle\frac{h^2}{2m}\left(\displaystyle\frac{3N}{8\pi V}\right)^{2/3}
            \]
        }
    \end{itemize}
\end{itemize}

\textbf{Applications:}
Used to study electrons in metals, semiconductors, and the behavior of neutron stars.
}



\subsection{Degeneracy Pressure}
\conceptbox{
\textbf{Degeneracy Pressure:}
Prevents the collapse of fermionic systems due to quantum mechanical effects, critical in white dwarfs and neutron stars.

\textbf{Key Relation:}
\eqbox{
\[
P \propto \left(\frac{\hbar^2}{m}\right)\left(\frac{N}{V}\right)^{5/3}.
\]
}
}

\newpage

\section{Quick Reference}


\subsection{Summary of Equations}

\begin{itemize}
    \tipbox{
    \textbf{Special Relativity:}
    \item Time Dilation: 
    \[
    \Delta t = \gamma \Delta t_0, \quad \gamma = \frac{1}{\sqrt{1 - v^2/c^2}}
    \]
    \item Length Contraction:
    \[
    L = L_0 \sqrt{1 - v^2/c^2}
    \]
    \item Relativistic Velocity Addition:
    \[
    u' = \frac{u + v}{1 + uv/c^2}
    \]
    \item Invariant Interval:
    \[
    \Delta s^2 = (c\Delta t)^2 - (\Delta x)^2 - (\Delta y)^2 - (\Delta z)^2
    \]
    \item Energy-Momentum Relation:
    \[
    E^2 = (pc)^2 + (m_0c^2)^2
    \]
    \item Four-Momentum Conservation:
    \[
    \sum p^\mu_{\text{initial}} = \sum p^\mu_{\text{final}}
    \]
    }

    \tipbox{
    \textbf{Quantum Theory of Light: }
    \item Planck's Law:
    \[
    I(\lambda, T) = \frac{2\pi hc^2}{\lambda^5}\frac{1}{e^{hc/\lambda kT} - 1}
    \]
    \item Einstein's Photoelectric Equation:
    \[
    hf = \phi + KE_{max}
    \]
    \item Threshold Frequency:
    \[
    hf_c = \phi
    \]
    \item Stopping Potential:
    \[
    KE_{max} = eV_s
    \]
    \item Compton Wavelength Shift:
    \[
    \lambda' - \lambda = \frac{h}{m_ec}(1 - \cos\theta)
    \]
    \item Energy Relation in Compton Scattering:
    \[
    \frac{1}{E'} - \frac{1}{E} = \frac{1 - \cos\theta}{m_ec^2}
    \]
    \item De Broglie Wavelength:
    \[
    \lambda = \frac{h}{p} = \frac{h}{mv}
    \]
    \item Wave Number:
    \[
    k = \frac{2\pi}{\lambda}
    \]
    }

    \tipbox{
    \textbf{Heisenberg Uncertainty Principle: }
    \item Position-Momentum Uncertainty:
    \[
    \Delta x \Delta p \geq \frac{\hbar}{2}
    \]
    \item Energy-Time Uncertainty:
    \[
    \Delta E \Delta t \geq \frac{\hbar}{2}
    \]
    \item Variance-Squared Average Relationship:
    \[
    \Delta A = \sqrt{\langle A^2 \rangle - \langle A \rangle^2}
    \]
    }

    \tipbox{
    \textbf{Schrödinger Equation: }
    \item 1D Time-Independent Schrödinger Equation:
    \[
    -\frac{\hbar^2}{2m} \frac{d^2\psi(x)}{dx^2} + V(x)\psi(x) = E\psi(x)
    \]
    \item 3D Schrödinger Equation (Spherical Coordinates):
    \[
    -\frac{\hbar^2}{2m} \nabla^2 \psi(r, \theta, \phi) + V(r)\psi(r, \theta, \phi) = E\psi(r, \theta, \phi)
    \]
    \item Wave function Normalization:
    \[
    \psi_{\text{norm}}(x) = A\psi(x), \quad A = \frac{1}{\sqrt{\int |\psi(x)|^2 dx}}.
    \]
    \item Particle in a Box Wave Function Normalized Solution 1D:
    \[
    \psi_n(x) = \sqrt{\displaystyle\frac{2}{L}} \sin(\displaystyle\frac{n\pi x}{L})
    \]
    \item Particle in a Box Wave Function Normalized Solution 3D:
    \[
    \psi_{n_1, n_2, n_3}(x, y, z) = \left(\displaystyle\frac{2}{L}\right)^{3/2}\sin(\displaystyle\frac{n_1\pi x}{L})\sin(\displaystyle\frac{n_2\pi y}{L})\sin(\displaystyle\frac{n_3\pi z}{L})
    \]
    \item Particle in a Box Energy Levels 1D:
    \[
    E_n = \frac{n^2\pi^2\hbar^2}{2mL^2}
    \]
    \item Particle in a Box Energy Levels 3D:
    \[
    E_{n_1, n_2, n_3} = \frac{\pi^2\hbar^2}{2mL^2}(n_1^2 + n_2^2 + n_3^2)
    \]
    \item Step Potential Reflection and Transmission Coefficients:
    \[
    R = \left(\frac{k_0 - k_1}{k_0 + k_1}\right)^2, \quad T = \frac{4k_0k_1}{(k_0 + k_1)^2}
    \]
    }

    \tipbox{
    \textbf{Statistical Mechanics: }
    \item Maxwell-Boltzmann Distribution:
    \[
    f_{MB}(E) = Ae^{-E/kT}
    \]
    \item Bose-Einstein Distribution:
    \[
    f_{BE}(E) = \frac{1}{e^\alpha e^{E/kT} - 1}
    \]
    \item Fermi-Dirac Distribution:
    \[
    f_{FD}(E) = \frac{1}{e^\alpha e^{E/kT} + 1}
    \]
    \item Occupation Number:
    \[
    n(E)dE = f(E)g(E)dE
    \]
    \item Critical Temperature for Bose-Einstein Condensation:
    \[
    T_c = \left(\frac{h^2}{2\pi mk}\right) \left(\frac{N}{V\zeta(3/2)}\right)^{2/3}
    \]
    \item Fraction in Ground State for Bose-Einstein Condensation:
    \[
    \frac{N_0}{N} = 1 - \left(\frac{T}{T_c}\right)^{3/2}
    \]
    \item Fermi Energy:
    \[
    E_F = \frac{h^2}{2m}\left(\frac{3N}{8\pi V}\right)^{2/3}
    \]
    }

    \tipbox{
    \textbf{Constants: }
    \item Stefan-Boltzmann Law:
    \[
    P = \sigma T^4, \quad \sigma = 5.67\times10^{-8} \, \text{W/m}^2\cdot\text{K}^4
    \]
    \item Wien's Displacement Law:
    \[
    \lambda_{\text{max}} = \frac{b}{T}, \quad b \approx 2.897\times10^{-3} \, \text{m}\cdot\text{K}
    \]
    }
\end{itemize}


\subsection{Mathematical Tools}
\begin{itemize}
    \tipbox{
    \item Trigonometric Identities:
    \begin{align*}
        &\cos^2x = \frac{1 + \cos(2x)}{2} \\
        &\sin^2x = \frac{1 - \cos(2x)}{2} \\
        &\sin(2x) = 2\sin x\cos x
    \end{align*}
    }
    \tipbox{
    \item Series: 
    \begin{itemize}
        \item Taylor Series Expansion: $\displaystyle\sum_{n = 0}^{\infty} C_n(x-a)^n$ where $ C_n = \displaystyle\frac{f^{(n)}(a)}{n!} $
        \item Binomial Series Expansion: $ (1 \pm x)^r = 1 \pm rx \pm ... $
    \end{itemize}
    }
    \tipbox{
    \item Integration Tips:
    \begin{itemize}
        \item For even functions: $\int_{-a}^a f(x)dx = 2\int_0^a f(x)dx$
        \item For odd functions: $\int_{-a}^a f(x)dx = 0$
        \item Gaussian integrals often appear in wave packets
    \end{itemize}
    }

    \tipbox{
    \item Complex Numbers Review
    \begin{itemize}
        \item Complex conjugate: $z^* = a - bi$ for $z = a + bi$
        \item Modulus: $|z| = \sqrt{a^2 + b^2}$
        \item Euler's formula: $e^{i\theta} = \cos\theta + i\sin\theta$
        \item Argument: $\arg(z) = \tan^{-1}\left(\displaystyle\frac{y}{x}\right)$
    \end{itemize}
    }

    \tipbox{
        \item Second Order Differential Equations
        \begin{itemize}
            \item If ODE has this form: $\displaystyle\frac{d^2 f}{dx^2} = k^2 f$ where k is a constant

            The solution is:
            $$ f(x) = Ae^{kx} + Be^{-kx} $$
        \end{itemize}
    }
\end{itemize}

\subsection{Important Constants}
\tipbox{
\begin{itemize}
    \item Stefan-Boltzmann constant: $\sigma = \stefan$
    \item Planck's constant: $h = \planckconst = 4.136\times10^{-15}\text{ eV}\cdot\text{s}$
    \item Reduced Planck's constant: $\hbar = \redplanck$
    \item Electron mass: $m_e = \electronmass$
    \item Speed of light: $c = \lightspeed$
    \item Bohr radius: $a_0 = \bohrradius$
    \item Compton wavelength of the electron: $\displaystyle\frac{h}{m_ec} = \compton$
\end{itemize}
}

\subsection{Unit Conversions}
\tipbox{
\begin{itemize}
    \item $1 \text{ eV} = 1.602 \times 10^{-19} \text{ J}$
    \item $hc = 1240$ eV $\cdot$ nm
    \item $\hbar c = \displaystyle\frac{1240}{2\pi}$ eV $\cdot$ nm
    \item Temperature: Convert between Kelvin and energy using $k_B = 1.38 \times 10^{-23} \text{ J/K}$.
\end{itemize}
}

\end{document}
