\documentclass{article}
\usepackage[a4paper, total={6in, 8in}]{geometry}
\usepackage{amsmath}
\usepackage{amssymb}
\usepackage{physics}
\usepackage{mathtools}
\usepackage{enumitem}
\usepackage{tcolorbox}
\usepackage{graphicx}
\usepackage{setspace}
\usepackage{siunitx}
\usepackage{tikz}
\usepackage{pgfplots}
\usepackage{float}

\usepackage{hyperref}
\hypersetup{
    colorlinks=true,
    linkcolor=black,
    filecolor=magenta,      
    urlcolor=blue,
    pdftitle={Modern Physics Final Study Guide},
    pdfpagemode=FullScreen,
}

\usepackage{fancyhdr}
\pagestyle{fancy}
\fancyhf{} % clear all header and footer fields
\fancyhead[L]{\leftmark} % put section name on the left
\fancyhead[R]{\thepage} % page number on the right
\renewcommand{\headrulewidth}{0.4pt} % horizontal line under header
\renewcommand{\sectionmark}[1]{\markboth{\small#1}{}}

% To handle the first page differently
\fancypagestyle{plain}{
    \fancyhf{} % clear all header and footer fields
    \renewcommand{\headrulewidth}{0pt} % no line for first page
}

\onehalfspacing

\pgfplotsset{compat=1.18}
\usetikzlibrary{angles,quotes,arrows.meta}

\AtBeginDocument{\RenewCommandCopy\qty\SI}

% Custom commands for consistent formatting
\newcommand{\eqbox}[1]{\begin{tcolorbox}[colback=gray!10] #1 \end{tcolorbox}}
\newcommand{\conceptbox}[1]{\begin{tcolorbox}[colback=blue!10] #1 \end{tcolorbox}}
\newcommand{\tipbox}[1]{\begin{tcolorbox}[colback=green!10] #1 \end{tcolorbox}}
\newcommand{\warningbox}[1]{\begin{tcolorbox}[colback=red!10] #1 \end{tcolorbox}}

% Constants
\newcommand{\planckconst}{6.626\times 10^{-34}\,\text{J}\cdot\text{s}}
\newcommand{\redplanck}{1.055\times 10^{-34}\,\text{J}\cdot\text{s}}
\newcommand{\electronmass}{9.11\times 10^{-31}\,\text{kg}}
\newcommand{\lightspeed}{3.0\times 10^{8}\,\text{m}/\text{s}}
\newcommand{\bohrradius}{0.0529\times 10^{-9}\,\text{m}}
\newcommand{\stefan}{5.67\times10^{-8}\text{ W/m}^2\cdot K^4}
\newcommand{\compton}{0.002426\text{ nm}}

\title{Modern Physics Final Study Guide}
\author{PHYS-122}
\date{Fall 2024}

\begin{document}
\maketitle

\newpage

\tableofcontents

\newpage

\section{Special Relativity}

\subsection{Key Concepts}

\conceptbox{
\textbf{Postulates:} 
\begin{itemize}
    \item Laws of physics are the same in all inertial frames.
    \item The speed of light is constant in a vacuum for all observers, regardless of their motion relative to the source.
\end{itemize}
}

\conceptbox{
\textbf{Relativity of Simultaneity:} 
Two events that appear simultaneous to one observer may not appear simultaneous to another observer moving relative to the first.
}

\conceptbox{
\textbf{Lorentz Transformations:}
The Lorentz transformations relate the spacetime coordinates of an event as measured in two inertial reference frames moving at a constant velocity relative to each other. These transformations preserve the invariant interval (\( \Delta s^2 \)).

\textbf{Transformations:}
\begin{itemize}
    \item For a frame \( S' \) moving with velocity \( v \) along the \( x \)-axis relative to \( S \):
    \begin{itemize}
        \item Time: \( t' = \gamma \left( t - \frac{vx}{c^2} \right) \)
        \item Space (\( x \)-direction): \( x' = \gamma \left( x - vt \right) \)
        \item Space (\( y \)- and \( z \)-directions): \( y' = y, \quad z' = z \)
    \end{itemize}
    \item \( \gamma = \frac{1}{\sqrt{1 - v^2/c^2}} \) is the Lorentz factor.
\end{itemize}

\textbf{Lorentz Transformation Matrix:}
The Lorentz transformation can be expressed using matrix multiplication:
\[
x' = \Lambda x,
\]
where \( \Lambda \) is the Lorentz transformation matrix. For a boost along the \( x \)-axis, \( \Lambda \) is:
\[
\Lambda =
\begin{bmatrix}
\gamma & -\gamma \beta & 0 & 0 \\
-\gamma \beta & \gamma & 0 & 0 \\
0 & 0 & 1 & 0 \\
0 & 0 & 0 & 1
\end{bmatrix},
\]
with \( \beta = v/c \).

\textbf{Key Properties:}
\begin{itemize}
    \item The determinant of \( \Lambda \) is \( +1 \), ensuring the transformations preserve spacetime orientation.
    \item The inverse transformation is obtained by negating \( v \), which flips the sign of \( \beta \):
    \[
    \Lambda^{-1} = \begin{bmatrix}
    \gamma & \gamma \beta & 0 & 0 \\
    \gamma \beta & \gamma & 0 & 0 \\
    0 & 0 & 1 & 0 \\
    0 & 0 & 0 & 1
    \end{bmatrix}.
    \]
    \item The Lorentz transformations preserve the invariant interval:
    \[
    \Delta s^2 = x \cdot x = x'\cdot x'.
    \]
\end{itemize}

\textbf{Applications:}
\begin{itemize}
    \item Lorentz transformations explain time dilation, length contraction, and the relativity of simultaneity.
    \item They are fundamental in deriving the transformations for four-vectors such as four-momentum and four-velocity.
\end{itemize}
}


\conceptbox{
\textbf{Relativistic Velocity Addition:} 
These problems typically involve three objects:
\begin{itemize}
    \item $S$: Stationary reference frame.
    \item $S'$: Reference frame of the second moving object.
    \item $u$: Velocity of the first object relative to the stationary frame.
    \item $v$: Velocity of the second object relative to the stationary frame.
    \item $u'$: Velocity of the first object relative to the second moving object.
\end{itemize}
}

\conceptbox{
\textbf{The Relativistic Invariant Interval:} 
A spacetime quantity that remains unchanged under Lorentz transformations.

\begin{itemize}
    \item \textbf{Properties:}
    \begin{itemize}
        \item Negative spatial components distinguish it from a Euclidean interval.
        \item It can be positive, negative, or zero:
        \begin{itemize}
            \item \textbf{Time-like ($\Delta s^2 > 0$):} Events can be connected by a physical particle moving slower than the speed of light.
            \item \textbf{Space-like ($\Delta s^2 < 0$):} Events are separated by more distance than time; no signal can connect them.
            \item \textbf{Light-like ($\Delta s^2 = 0$):} Events are connected by a light signal.
        \end{itemize}
    \end{itemize}
    \item \textbf{Relationship to Proper Time and Proper Length:}
    \begin{itemize}
        \item \textbf{Proper Time ($\Delta \tau$):} In the frame where the particle is stationary, the interval relates to proper time.
        \item \textbf{Proper Length ($\Delta L$):} In the frame where events are simultaneous, the interval relates to proper length.
    \end{itemize}
\end{itemize}
}

\conceptbox{
\textbf{Four-Vectors:} A four-vector in special relativity is a quantity defined in four-dimensional spacetime that transforms under Lorentz transformations. It has the general form:
\[
a = (a^0, \vec{a}),
\]
where \( a^0 \) is the time component ($c\Delta t$), and \( \vec{a} = (a^1, a^2, a^3) \) is the spatial 3-vector component.


\textbf{Key Properties:}
\begin{itemize}
    \item The \textbf{invariant dot product} of two four-vectors \( a \) and \( b \) is:
    \[
    a \cdot b = a^0 b^0 - \vec{a} \cdot \vec{b}.
    \]
    This quantity is Lorentz invariant (unchanged under Lorentz transformations).
    \item The invariant magnitude of a four-vector is a special case of the dot product:
    \[
    a \cdot a = (a^0)^2 - |\vec{a}|^2.
    \]
    \item The magnitude classifies four-vectors as:
    \begin{itemize}
        \item \textbf{Time-like:} \( a \cdot a > 0 \) then $ \Delta s^2 = c^2\tau^2$.
        \item \textbf{Space-like:} \( a \cdot a < 0 \) then $ -\Delta s^2 = L $.
        \item \textbf{Light-like:} \( a \cdot a = 0 \) then $ \Delta s^2 = 0 $.
    \end{itemize}
\end{itemize}

\textbf{Common Examples:}
\begin{itemize}
    \item \textbf{Four-Position:} \( x = (ct, \vec{x}) \), where \( ct \) represents the time component and \( \vec{x} = (x, y, z) \) is the spatial position.
    \item \textbf{Four-Velocity:} \( u = \displaystyle\frac{dx}{d\tau} = \gamma (c, \vec{v}) \), where \( \gamma = \frac{1}{\sqrt{1 - v^2/c^2}} \), \( c \) is the speed of light, and \( \vec{v} \) is the 3-velocity.
    \item \textbf{Four-Momentum:} \( p = mu = \gamma m (c, \vec{v}) = \left( \frac{E}{c}, \vec{p} \right) \), where \( E \) is the total energy and \( \vec{p} \) is the 3-momentum.
\end{itemize}

\textbf{Applications of the Invariant Dot Product:}
\begin{itemize}
    \item \textbf{Four-Velocity:} The invariant magnitude of \( u \) is always positive and equal to \( c^2 \):
    \[
    u \cdot u = c^2.
    \]
    \item \textbf{Four-Momentum:} The invariant magnitude of \( p \) relates to the particle's rest mass \( m \):
    \[
    p \cdot p = m^2 c^2.
    \]
    For massless particles, like photons, \( p \cdot p = 0 \) and \( E = |\vec{p}|c \).
\end{itemize}
}



\conceptbox{
\textbf{Relativistic Conservation Laws:} 
The principles of conservation of energy and momentum extend to special relativity with some key differences from classical mechanics. 

\begin{itemize}
    \item \textbf{Relativistic Total Energy:} 
    \begin{itemize}
        \item Total energy includes kinetic energy and rest energy: \( E = \gamma m_0 c^2 \), where \( \gamma = \frac{1}{\sqrt{1 - v^2/c^2}} \).
        \item Rest energy (\( m_0 c^2 \)) accounts for the energy associated with the rest mass of a particle.
        \item Total energy is conserved in all relativistic processes, including collisions.
    \end{itemize}

    \item \textbf{Relativistic Momentum:} 
    \begin{itemize}
        \item Momentum in special relativity is defined as \( \vec{p} = \gamma m_0 \vec{v} \).
        \item The magnitude and direction of the total momentum in an isolated system are conserved.
    \end{itemize}

    \item \textbf{Relativistic Collisions:} 
    \begin{itemize}
        \item Unlike classical mechanics, all forms of energy (including rest energy) contribute to the total energy of the system.
        \item The distinction between elastic and inelastic collisions in classical mechanics is less useful in relativity. Instead, we track the conservation of total energy and momentum.
        \item Mass is not generally conserved in relativistic collisions. For example, mass can be converted into energy (e.g., pair annihilation) or vice versa (e.g., particle creation).
    \end{itemize}

    \item \textbf{Implications:} 
    \begin{itemize}
        \item Energy and momentum form components of the four-momentum vector: \( p^\mu = \left( \frac{E}{c}, \vec{p} \right) \).
        \item Conservation laws apply to the four-momentum vector in all inertial frames:
        \[
        \sum p^\mu_{\text{initial}} = \sum p^\mu_{\text{final}}
        \]
        \item These conservation laws ensure the invariance of physical processes across reference frames.
    \end{itemize}
\end{itemize}
}

\subsection{Essential Equations}
\eqbox{
\textbf{Time Dilation:} $\Delta t = \gamma \Delta t_0$, where $\gamma = \frac{1}{\sqrt{1 - v^2/c^2}}$.
}

\eqbox{
\textbf{Length Contraction:} $L = L_0 \gamma^{-1} = L_0 \sqrt{1 - v^2/c^2}$.
}

\eqbox{
\textbf{Relativistic Addition of Velocities:} $u' = \displaystyle\frac{u + v}{1 + uv/c^2}$.
}

\eqbox{
\textbf{Lorentz Transformations:}
\begin{itemize}
    \item \textbf{Time:} $t' = \gamma(t - \frac{vx}{c^2})$.
    \item \textbf{Space:} $x' = \gamma(x - vt)$.
\end{itemize}
}

\eqbox{
\textbf{Relativistic Dynamics:}
\begin{itemize}
    \item \textbf{Kinetic Energy:} $KE = (\gamma - 1)m_0c^2$.
    \item \textbf{Total Energy:} $E = \gamma m_0c^2$.
    \item \textbf{Rest Mass: } $E = m_0c^2$
\end{itemize}
}

\eqbox{
\textbf{Energy-Momentum Relation:} $E^2 = (pc)^2 + (m_0c^2)^2$.
}

\eqbox{
\textbf{Four-Position:}
\begin{itemize}
    \item Defined as $x = (c\Delta t, x, y, z)$, where $x$ is the spacetime coordinate.
    \item Invariant interval: $\Delta s^2 = (c \Delta t)^2 - \Delta x^2 -  \Delta y^2 - \Delta z^2$.
\end{itemize}
}

\eqbox{
\textbf{Four-Velocity:}
\begin{itemize}
    \item Defined as $u^\mu = \frac{dx^\mu}{d\tau}$, where $\tau$ is proper time.
    \item Invariant dot product: $u^\mu u_\mu = c^2$.
\end{itemize}
}

\eqbox{
\textbf{Four-Momentum:}
\begin{itemize}
    \item Defined as $p = m_0 u$, where $m_0$ is the rest mass.
    \item Invariant: $p \cdot p = m_0^2c^2$.
\end{itemize}
}
\section{Early Quantum Theory and Quantum Mechanics}
\conceptbox{
\textbf{Quantum Theory of Light:}
\begin{itemize}
    \item \textbf{Blackbody Problem and Ultraviolet Catastrophe:}
    Classical physics predicted infinite energy at short wavelengths (ultraviolet catastrophe). Planck resolved this by introducing quantized energy levels \( E = nhf \), leading to Planck's radiation law.
    
    \item \textbf{Photoelectric Effect:}
    Light behaves as particles (photons). The energy of emitted electrons depends on the frequency, not intensity, of the incident light:
    \[
    E = hf - \phi,
    \]
    where \( \phi \) is the work function.

    \item \textbf{Compton Scattering:}
    Photons scatter off electrons, resulting in a wavelength shift:
    \[
    \Delta \lambda = \frac{h}{m_e c} (1 - \cos\theta),
    \]
    confirming the particle nature of light.

    \item \textbf{Wave-Particle Duality:}
    Light and matter exhibit both wave and particle properties. De Broglie hypothesized the wavelength of matter waves:
    \[
    \lambda = \frac{h}{p}.
    \]

    \item \textbf{Emission and Absorption Spectra:}
    Atoms absorb/emits photons corresponding to energy level transitions:
    \[
    \Delta E = hf.
    \]

    \item \textbf{Bohr Model:}
    Electrons occupy quantized orbits, with energy levels:
    \[
    E_n = -\frac{13.6}{n^2} \text{ eV for hydrogen}.
    \]

    \item \textbf{Franck-Hertz Experiment:}
    Confirmed quantized energy levels by measuring discrete energy loss during electron collisions with atoms.

    \item \textbf{Heisenberg Uncertainty Principle:}
    Fundamental limits on measurement precision:
    \[
    \Delta x \Delta p \geq \frac{\hbar}{2}, \quad \Delta E \Delta t \geq \frac{\hbar}{2}.
    \]

    \item \textbf{Group and Phase Velocities:}
    \[
    v_g = \frac{d\omega}{dk}, \quad v_p = \frac{\omega}{k}.
    \]
    Group velocity corresponds to the velocity of the particle, while phase velocity is associated with the wave motion.
\end{itemize}
}

\conceptbox{
\textbf{Schrödinger Equation:} Governs the quantum mechanical behavior of particles.
\begin{itemize}
    \item \textbf{1D Time-Independent Schrödinger Equation:}
    \[
    -\frac{\hbar^2}{2m} \frac{d^2\psi(x)}{dx^2} + V(x)\psi(x) = E\psi(x).
    \]

    \item \textbf{3D Schrödinger Equation (spherical coordinates):}
    \[
    -\frac{\hbar^2}{2m} \nabla^2 \psi(r, \theta, \phi) + V(r)\psi(r, \theta, \phi) = E\psi(r, \theta, \phi).
    \]
    The solution is separable:
    \[
    \psi(r, \theta, \phi) = R(r)Y(\theta, \phi),
    \]
    where \( Y(\theta, \phi) \) are spherical harmonics.

    \item \textbf{Free Particle:}
    For \( V(x) = 0 \), solutions are plane waves:
    \[
    \psi(x) = Ae^{i(kx - \omega t)}.
    \]

    \item \textbf{Particle in a Box:}
    For \( V(x) = 0 \) inside the box and \( V(x) = \infty \) outside:
    \[
    \psi_n(x) = \sqrt{\frac{2}{L}} \sin\left(\frac{n\pi x}{L}\right), \quad E_n = \frac{n^2\pi^2\hbar^2}{2mL^2}.
    \]

    \item \textbf{Step Potential:}
    \begin{itemize}
        \item \( E < U \): Tunneling occurs. The wave function decays exponentially in the classically forbidden region.
        \item \( E > U \): The wave function is continuous, with partial reflection and transmission.
    \end{itemize}
\end{itemize}
}

\conceptbox{
\textbf{Hydrogen Atom:} Solutions to the Schrödinger equation in spherical coordinates yield quantized energy levels and quantum numbers.

\begin{itemize}
    \item \textbf{Energy Levels:} 
    \[
    E_n = -\frac{13.6 \, \text{eV}}{n^2}.
    \]

    \item \textbf{Quantum Numbers:}
    \begin{itemize}
        \item Principal quantum number \( n \): Determines energy level (\( n = 1, 2, 3, \dots \)).
        \item Orbital angular momentum quantum number \( l \): Determines the shape of the orbital (\( l = 0, 1, \dots, n-1 \)).
        \item Magnetic quantum number \( m_l \): Determines the orientation (\( m_l = -l, -l+1, \dots, l \)).
        \item Spin quantum number \( m_s \): Intrinsic angular momentum (\( m_s = \pm \frac{1}{2} \)).
    \end{itemize}

    \item \textbf{Wave Functions:}
    \[
    \psi(r, \theta, \phi) = R_{nl}(r)Y_l^m(\theta, \phi),
    \]
    where \( R_{nl}(r) \) is the radial wave function, and \( Y_l^m(\theta, \phi) \) are spherical harmonics.

    \item \textbf{Radial Probability Distribution:}
    The probability density \( P(r) \) is given by:
    \[
    P(r) = r^2|R_{nl}(r)|^2.
    \]
\end{itemize}
}
\subsection{Essential Equations}

\eqbox{
\textbf{Planck's Hypothesis:} \( E = nhf \).
}

\eqbox{
\textbf{Photoelectric Effect:} \( KE_{\text{max}} = hf - \phi \).
}

\eqbox{
\textbf{Compton Scattering:} \( \Delta \lambda = \frac{h}{m_e c} (1 - \cos\theta) \).
}

\eqbox{
\textbf{De Broglie Wavelength:} \( \lambda = \frac{h}{p} \).
}

\eqbox{
\textbf{Heisenberg Uncertainty Principles:} \( \Delta x \Delta p \geq \frac{\hbar}{2}, \quad \Delta E \Delta t \geq \frac{\hbar}{2} \).
}

\eqbox{
\textbf{Particle in a Box:} \( E_n = \displaystyle\frac{n^2\pi^2\hbar^2}{2mL^2}, \quad \psi_n(x) = \sqrt{\frac{2}{L}} \sin\left(\frac{n\pi x}{L}\right) \).
}

\eqbox{
\textbf{Radial Probability Distribution:} \( P(r) = r^2|R_{nl}(r)|^2 \).
}

\section{Statistical Mechanics}

\subsection{Key Concepts}
\conceptbox{
\begin{itemize}
    \item Microstates and macrostates: Connection between microscopic and macroscopic descriptions.
    \item Boltzmann distribution: $P_i = \frac{e^{-E_i/k_BT}}{Z}$.
    \item Partition function: Summation of states to understand thermodynamic properties, $Z = \sum_i e^{-E_i/k_BT}$.
    \item Occupation numbers: Average number of particles in a state, $n_i = \frac{1}{e^{(E_i-\mu)/k_BT} \pm 1}$ (for fermions and bosons).
    \item Degeneracy: The number of microstates corresponding to a single energy level.
    \item Distinguishable vs. indistinguishable particles: Boltzmann distribution for distinguishable particles; Fermi-Dirac and Bose-Einstein distributions for indistinguishable particles.
\end{itemize}
}

\subsection{Essential Equations}
\eqbox{
\begin{align*}
    &\text{Boltzmann distribution: } P_i = \frac{e^{-E_i/k_BT}}{Z} \\
    &\text{Partition function: } Z = \sum_i g_i e^{-E_i/k_BT} \\
    &\text{Fermi-Dirac distribution: } n_i = \frac{1}{e^{(E_i-\mu)/k_BT} + 1} \\
    &\text{Bose-Einstein distribution: } n_i = \frac{1}{e^{(E_i-\mu)/k_BT} - 1} \\
    &\text{Average energy: } \langle E \rangle = \sum_i P_i E_i = -\frac{\partial \ln Z}{\partial \beta} \\
    &\text{Probability of a state: } P(E) = \frac{g_i e^{-E/k_BT}}{Z}
\end{align*}
}

\subsection{Units and Unit Conversions}
\tipbox{
\begin{itemize}
    \item Energy: $1 \text{ eV} = 1.602 \times 10^{-19} \text{ J}$.
    \item Temperature: Convert between Kelvin and energy using $k_B = 1.38 \times 10^{-23} \text{ J/K}$.
    \item Wavelength and frequency: $E = hf$, $\lambda = \frac{c}{f}$, with $hc = 1240 \text{ eV} \cdot \text{nm}$.
\end{itemize}
}

\section{Quick Reference}

\subsection{Mathematical Tools}
\begin{itemize}
    \tipbox{
    \item Trigonometric Identities:
    \begin{align*}
        &\cos^2x = \frac{1 + \cos(2x)}{2} \\
        &\sin^2x = \frac{1 - \cos(2x)}{2} \\
        &\sin(2x) = 2\sin x\cos x
    \end{align*}
    }
    \tipbox{
    \item Series: 
    \begin{itemize}
        \item Taylor Series Expansion: $\displaystyle\sum_{n = 0}^{\infty} C_n(x-a)^n$ where $ C_n = \displaystyle\frac{f^{(n)}(a)}{n!} $
        \item Binomial Series Expansion: $ (1 \pm x)^r = 1 \pm rx \pm ... $
    \end{itemize}
    }
    \tipbox{
    \item Integration Tips:
    \begin{itemize}
        \item For even functions: $\int_{-a}^a f(x)dx = 2\int_0^a f(x)dx$
        \item For odd functions: $\int_{-a}^a f(x)dx = 0$
        \item Gaussian integrals often appear in wave packets
    \end{itemize}
    }

    \tipbox{
    \item Complex Numbers Review
    \begin{itemize}
        \item Complex conjugate: $z^* = a - bi$ for $z = a + bi$
        \item Modulus: $|z| = \sqrt{a^2 + b^2}$
        \item Euler's formula: $e^{i\theta} = \cos\theta + i\sin\theta$
        \item Argument: $\arg(z) = \tan^{-1}\left(\displaystyle\frac{y}{x}\right)$
    \end{itemize}
    }

    \tipbox{
        \item Second Order Differential Equations
        \begin{itemize}
            \item If ODE has this form: $\displaystyle\frac{d^2 f}{dx^2} = k^2 f$ where k is a constant

            The solution is:
            $$ f(x) = Ae^{kx} + Be^{-kx} $$
        \end{itemize}
    }
\end{itemize}

\subsection{Important Constants}
\tipbox{
\begin{itemize}
    \item Stefan-Boltzmann constant: $\sigma = \stefan$
    \item Planck's constant: $h = \planckconst = 4.136\times10^{-15}\text{ eV}\cdot\text{s}$
    \item Reduced Planck's constant: $\hbar = \redplanck$
    \item Electron mass: $m_e = \electronmass$
    \item Speed of light: $c = \lightspeed$
    \item Bohr radius: $a_0 = \bohrradius$
    \item Compton wavelength of the electron: $\displaystyle\frac{h}{m_ec} = \compton$
\end{itemize}
}

\subsection{Unit Conversions}
\tipbox{
\begin{itemize}
    \item $1 \text{ eV} = 1.602 \times 10^{-19} \text{ J}$
    \item $hc = 1240$ eV $\cdot$ nm
    \item $\hbar c = \displaystyle\frac{1240}{2\pi}$ eV $\cdot$ nm
\end{itemize}
}

\end{document}
